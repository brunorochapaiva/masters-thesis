\documentclass[11pt % Font size
              ]{article}
\usepackage[ a4paper % Paper size and format
           , onecolumn % Number of columns: onecolumn or twocolumn
           , total={6in, 8in} % Text area: {width, height}
           ]{geometry}

\title{A Model Theoretic Study of Allen's Interval Algebra}
\author{Bruno da Rocha Paiva - bmd18}
\date{} % Date of writing: empty, today or 25.12.00

\usepackage{amsmath}
\usepackage{amsthm}
\usepackage{amssymb}
\usepackage{parskip}
\usepackage{tikz}
\usepackage{quiver}

\usepackage{subfiles}

% General Symbols
\newcommand{\N}{\mathbb{N}}
\newcommand{\Z}{\mathbb{Z}}
\newcommand{\Q}{\mathbb{Q}}
\newcommand{\R}{\mathbb{R}}

% Category Theory
\newcommand{\id}[1]{\text{id}_{#1}}
\newcommand{\unit}[1]{\eta_{#1}}
\newcommand{\counit}[1]{\epsilon_{#1}}

% Model Theory
\newcommand{\lang}{\mathcal{L}}
\newcommand{\theory}{\mathbb{T}}
\newcommand{\age}[1]{\text{Age}\left(#1\right)}

\newcommand{\finslo}{\textbf{FCh}}
\newcommand{\finaia}{\textbf{FIA}}


\newcommand{\lslo}{\lang_\text{SLO}}
\newcommand{\tslo}{\theory_\text{SLO}}

% Allen Interval Algebra Symbols
\newcommand{\laia}{\lang_\text{AIA}}
\newcommand{\taia}{\theory_\text{AIA}}
\newcommand{\istart}[1]{#1_{-}}
\newcommand{\iend}[1]{#1_{+}}

% Allen Interval Algebra Relations
\newcommand{\aiaarrow}[1]{\overset{#1}{\longrightarrow}}
\newcommand{\before}{\overset{<}{\longrightarrow}}
\newcommand{\meets}{\overset{m}{\longrightarrow}}
\newcommand{\overlaps}{\overset{o}{\longrightarrow}}
\newcommand{\starts}{\overset{s}{\longrightarrow}}
\newcommand{\finishes}{\overset{f}{\longrightarrow}}
\newcommand{\contained}{\overset{d}{\longrightarrow}}
\newcommand{\after}{\overset{<}{\longleftarrow}}
\newcommand{\metby}{\overset{m}{\longleftarrow}}
\newcommand{\overlappedby}{\overset{o}{\longleftarrow}}
\newcommand{\startedby}{\overset{s}{\longleftarrow}}
\newcommand{\finishedby}{\overset{f}{\longleftarrow}}
\newcommand{\contains}{\overset{d}{\longleftarrow}}

% Categories of models
\newcommand{\mods}[2]{\text{Mod}\left(#1,#2\right)}
\newcommand{\aias}{\mods{\laia}{\taia}}
\newcommand{\slos}{\mods{\lslo}{\tslo}}

% Intervals construction
\newcommand{\inter}[1][-]{\text{Int}\left(#1\right)}

% Points construction
\newcommand{\psim}{\sim}
\newcommand{\peq}{\phi_\sim}
\newcommand{\plt}{\phi_<}
\newcommand{\points}[1][-]{\text{Pts}\left(#1\right)}

% Environments
\theoremstyle{plain}
\newtheorem{thm}{Theorem}%[subsection]
\newtheorem{lemma}[thm]{Lemma}
\newtheorem{prop}[thm]{Proposition}
\newtheorem{cor}[thm]{Corollary}

\theoremstyle{definition}
\newtheorem{defn}[thm]{Definition}

\theoremstyle{remark}
\newtheorem*{rem}{Remark}

\begin{document}

\maketitle

\begin{defn}
  We define the language of strict linear orders $\lslo$ as the single binary relation $\{ < \}$.

  We define the theory of strict linear orders as
  \begin{align*}
    \tslo = \{ & \forall a, \lnot\ a < a, \\
              & \forall a, \forall b, \forall c, a < b \land b < c \rightarrow a < c \\
              & \forall a, \forall b, a < b \lor a = b \lor b < a \}
  \end{align*}
\end{defn}

\begin{defn}
  We define the language of Allen interval algebras $\laia$ as
  \begin{equation*}
    \laia = \{ \before, \meets, \overlaps, \starts, \finishes, \contained,
               \after, \metby, \overlappedby, \startedby, \finishedby, \contains \}
  \end{equation*}

  (Unsure on how to best write out the definition of $\taia$, especially with the "transitivity"
  axioms).
\end{defn}

% ------ Justifying this axiomatisation of interval algebras

\begin{defn}
  Given a linear order $L$ we define its set of non-zero intervals $\inter[P]$ as the set
  \begin{equation*}
    \inter[L] = \left\{(x_1,x_2)\ |\ x_1 < x_2 \right\} \subseteq L^2
  \end{equation*}
  We can turn this into a $\laia$-structure under the interpretations:
  \begin{itemize}
    \item $(x_1,x_2) \before (y_1,y_2)$ if $x_1<x_2<y_1<y_2$;
    \item $(x_1,x_2) \meets (y_1,y_2)$ if $x_1<x_2=y_1<y_2$;
    \item $(x_1,x_2) \overlaps (y_1,y_2)$ if $x_1<y_1<x_2<y_2$;
    \item $(x_1,x_2) \starts (y_1,y_2)$ if $x_1=y_1<x_2<y_2$;
    \item $(x_1,x_2) \finishes (y_1,y_2)$ if $y_1<x_1<x_2=y_2$;
    \item $(x_1,x_2) \contained (y_1,y_2)$ $y_1<x_1<x_2<y_2$;
    \item $(x_1,x_2) \overset{i}{\longleftarrow} (y_1,y_2)$ if and only if
      $(y_1,y_2) \overset{i}{\longrightarrow} (x_1,x_2)$ for $i \in \{<,m,o,s,f,d\}$.
  \end{itemize}
\end{defn}

\begin{thm}
  Given a strict linear order $L$, $\inter[L]$ is a model of $\taia$ under the above
  interpretations.
\end{thm}
\begin{proof}
  Tedious proof by cases goes here.
\end{proof}

\begin{cor}
  Allen's interval algebras are satisfiable
\end{cor}

The theory is satisfiable in a reasonable way, inducing the expected models from linear orders.
To further enforce this tight connection, we see interval algebras also induce a quite reasonable
linear order

\begin{defn}
  Given an Allen interval algebra $A$, we define $\points[A] = \frac{A + A}{\psim}$ where
  $\psim$ is an equivalence relation on the set $A + A$ defined by
  \begin{equation*}
    (n,I) \psim (m,J) \iff A \models \peq(n,m)(I,J)
  \end{equation*}
  Where $\peq$ assigns $\laia$-formulas with open variables $I,J$ to pairs $(n,m) \in \{0,1\}^2$
  \begin{equation*}
    \peq(n,m)(I,J) = \begin{cases}
      I \starts J \lor I \startedby J \lor I = J \quad & \text{if } n = m = 0\\
      I \finishes J \lor I \finishedby J \lor I = J    & \text{if } n = m = 1 \\
      I \metby J                                       & \text{if } n = 0,\ m = 1 \\
      I \meets J                                       & \text{if } n = 1,\ m = 0
    \end{cases}
  \end{equation*}
\end{defn}

\begin{thm}
  Given an Allen interval algebra $A$, the interpretation of the symbol $<$ in $\points[A]$ given by
  $(n,I) < (m,J) \iff A \models \plt(n,m)(I,J)$ is well-defined and turns $A$ into a model of
  $\tslo$.

  Here, $\plt$ is a function assigning $\laia$-formulas with open variables $I,J$ to pairs
  $(n,m) \in \{0,1\}^2$ defined as:
  \begin{equation*}
    \plt(n,m)(I,J) = \begin{cases}
      I \before J \lor I \meets J \lor I \overlaps J  \lor I \finishedby J \lor I \contains J
                                                & \text{if } n = m = 0\\
        I \before J \lor I \meets J \lor I \overlaps J \lor I \starts J \lor I \contained J
                                                & \text{if } n = m = 1 \\
      \lnot\,(I \after J) \land \lnot\,(I \metby J) \land \lnot\,(I \startedby J)
                                                & \text{if } n = 0,\ m = 1 \\
      I \before J                               & \text{if } n = 1,\ m = 0
    \end{cases}
  \end{equation*}
\end{thm}
\begin{proof}
  %TODO Show that $<$ is well defined

  Now we know that our definition of $<$ does not depend on a choice of representative, we need
  to check whether it satisfies $\tslo$:
  \begin{itemize}
    \item \textbf{irreflexivity}: Fix some element $[(n,I)] \in \points[A]$. Since the interval
      algebra relations (along with equality) are mutually exclusive and $I = I$, no other relation
      can hold for the pair $(I,I)$. Regardless of the value of $n$, $\plt(n,n)(I,I)$ does not
      include $I = I$ as a disjunct, so $A \not\models \plt(n,n)(I,I)$ and $<$ is irreflexive.
    \item \textbf{transitivity}:
    \item \textbf{trichotomy}:
  \end{itemize}
\end{proof}

%TODO: Consider adding table justifying these definitions? Something like:
%| I relation J | drawing of I and J | is I- < J- | is I+ < J+ | is I- < J + | is I+ < J - |

Now that this axiomatisation is justified, show that we can't remove any axioms without making
it into something we do not want.

% ------ A slight detour into category theory

\begin{defn}
  Given a theory $\theory$ over language $\lang$, we denote by $\mod{\lang}{\theory}$ the category
  with objects the models of $\theory$ and arrows the $\lang$-embeddings.
\end{defn}

\begin{thm}
  We can turn $\inter$ into a functor from $\mod{\lslo}{\tslo}$ into
  $\mod{\laia}{\taia}$ by sending arrows $f : M \to N$ in $\mod{\lslo}{\tslo}$ to the arrow
  $\inter[f] : \inter[M] \to \inter[N]$ in $\mod{\laia}{\taia}$ defined by
  \begin{equation*}
    \inter[f](x_1, x_2) = (f(x_1), f(x_2))
  \end{equation*}
\end{thm}
\begin{proof}
  Check the map preserves interpretations.

  Standard checking of functor axioms.
\end{proof}

\begin{thm}
  We can turn $\points$ into a functor from $\mod{\laia}{\taia}$ into
  $\mod{\lslo}{\tslo}$ by sending arrows $f : A \to B$ in $\mod{\laia}{\taia}$ to the arrow
  $\points[f] : \points[A] \to \points[B]$ in $\mod{\lslo}{\tslo}$ defined by
  \begin{equation*}
    \points[f](0, I) = (0, f(I)) \qquad\text{and}\qquad \points[f](1,I) = (1, f(I))
  \end{equation*}
\end{thm}
\begin{proof}
  Given an arrow $f : A \to B$, we must check that $\points[f]$ is a well defined map, and that it
  is an $\lslo$-embedding. These facts both follow by noticing that $f$ is an
  $\laia$-embedding, so it preserves the truth of quantifier-free $\laia$-formulas, so
  \begin{align*}
    (n,I) \psim (m,J)
      & \iff A \models \peq(n,m)(I,J) \\
      & \iff A \models \peq(n,m)(f(I),f(J)) \\
      & \iff (n,f(I)) \psim (m,f(J))
  \end{align*}
  and similarly
  \begin{align*}
    [(n,I)] < [(m,J)]
      & \iff A \models \plt(n,m)(I,J) \\
      & \iff A \models \plt(n,m)(f(I),f(J)) \\
      & \iff \points[f]([(n,I)]) < \points[f]([(m,J)])
  \end{align*}
  for any two $(n,I),(m,J) \in A + A$, since $\peq(n,m)$ and $\plt(n,m)$ are always quantifier-free.

  Next, to see that $\points$ satisfies the functor axioms:
  \begin{itemize}
    \item \textbf{preserves identity arrows}: Fix some interval algebra $A$ and some element
    $[(n, I)] \in \points[A]$. Then notice that
      \begin{equation*}
        \points[\id{A}]([(n,I)]) = [(n, \id{A}(I))] = [(n,I)] = \id{\points[A]}([(n,I)])
      \end{equation*}
      Hence $\points[\id{A}] = \id{\points[A]}$.
    \item \textbf{respects arrow composition}: Fix interval algebras $A,B,C$ along with arrows
      $f : A \to B$, $g : B \to C$. Then for all elements $[(n,I)] \in \points[A]$:
      \begin{align*}
        \points[g \circ f]([(n,I)])
          & = [(n,g \circ f (I))] = [(n, g(f(I)))] \\
          & = \points[g]\left( \points[f]([n,I]) \right) \\
          & = \points[g] \circ \points[f] ([n,I])
      \end{align*}
      Hebce $\points[g \circ f] = \points[g] \circ \points[f]$ as expected.
  \end{itemize}
\end{proof}

\begin{thm}
  $\points$ is left adjoint to $\inter$.
\end{thm}
\begin{proof}
  For an interval algebra $A$ and a strict linear order $L$, we have
  \begin{equation*}
    \text{Hom}(\points[A], L) \cong \text{Hom}(A, \inter[L])
  \end{equation*}
  A map of linear orders from the start/end points of A to L, enforces a mapping of interval
  algebras from A to the intervals of L, by checking where the start/end points of the interval end.

  The above map is "canonical" so it should be natural in A and L.

  Notice also that
  \begin{equation*}
    \text{Hom}(\inter[L],A) \not\cong \text{Hom}(L, \points[A])
  \end{equation*}
  Since A does not necessarily have all the intervals given by $\points[A]$, so if we have a map on
  the right, it isn't guaranteed that we have a corresponding map on the left.
\end{proof}

% ------ Homogeneous tings eh

\begin{thm}
  The class $\finslo$ of finite strict linear orders satisfies the hereditary property (HP),
  the joint embedding property (JEP), the amalgamation property (AP) and is essentially countable
  (EC). In other words, $\finslo$ is a Fraïssé class.
\end{thm}
\begin{proof}
  Something like
  \begin{itemize}
    \item HP: Show the result for all relational, universal theories, since this is referenced in
      the later proof.
    \item JEP: Given two finite strict linear orders, embed them into their disjoint union with
      the left order befored the right order
    \item AP: Given two finite strict linear orders A and B, with C embedding into both,
      embed A and B into their product ordered lexicographically
    \item EC: Show the result for all finite languages, since this is referenced in the later proof.
  \end{itemize}
\end{proof}

\begin{thm}
  The Fraïssé limit of $\finslo$ is $\Q$
\end{thm}
\begin{proof}
  Show that $\Q$ is homogeneous and that its age is $\finslo$.
\end{proof}

\begin{thm}
  The class $\finaia$ of finite interval algebras is a Fraïssé class.
\end{thm}
\begin{proof}
  First, notice that $\points$ must send finite interval algebras to finite strict linear orders.
  In fact, given an interval algebra $A$, $|\points[A]| \leq 2|A|$ since
  $\points[A]$ is a quotient of $A + A$. Similarly, $\inter$ must also send finite strict linear
  orders to finite interval algebras: given a strict linear order $L$, $|\inter[L]| = |L|^2$.

  Next, with this in mind, we check the Fraïssé class conditions:
  \begin{itemize}
    \item \textbf{HP}: This follows from the fact that $\laia$ is a relational language and
      $\taia$ is a universal theory, same as the case for strict linear orders.
    \item \textbf{JEP}: Given two finite interval algebras $A$ and $B$, we use the JEP of strict
      linear orders to get the following diagram in $\slos$
      \[\begin{tikzcd}
        {\points[A]} \\
        && \Omega \\
        {\points[B]}
        \arrow["f", from=1-1, to=2-3]
        \arrow["g"', from=3-1, to=2-3]
      \end{tikzcd}\]
      Then, applying $\inter$ and using the adjunction unit $\eta$, we get
      % https://q.uiver.app/?q=WzAsNSxbMiwwLCJcXGludGVyW1xccG9pbnRzW0FdXSJdLFsyLDIsIlxcaW50ZXJbXFxwb2ludHNbQl1dIl0sWzQsMSwiXFxpbnRlcltcXE9tZWdhXSJdLFswLDAsIkEiXSxbMCwyLCJCIl0sWzAsMiwiXFxpbnRlcltmXSJdLFsxLDIsIlxcaW50ZXJbZ10iLDJdLFs0LDEsIlxcdW5pdHtCfSIsMl0sWzMsMCwiXFx1bml0e0F9Il1d
      \[\begin{tikzcd}
        A && {\inter[\points[A]]} \\
        &&&& {\inter[\Omega]} \\
        B && {\inter[\points[B]]}
        \arrow["{\inter[f]}", from=1-3, to=2-5]
        \arrow["{\inter[g]}"', from=3-3, to=2-5]
        \arrow["{\unit{B}}"', from=3-1, to=3-3]
        \arrow["{\unit{A}}", from=1-1, to=1-3]
      \end{tikzcd}\]
      And the composites $\inter[f] \circ \unit{A}$ and $\inter[g] \circ \unit{B}$ along with
      the interval algebra $\Omega$ give us the joint embedding of $A$ and $B$.
    \item \textbf{AP}: Suppose we have the following diagram in $\aias$
      % https://q.uiver.app/?q=WzAsNCxbMiwwLCJBIl0sWzIsMiwiQiJdLFswLDEsIkMiXSxbNCwxXSxbMiwwLCJmIl0sWzIsMSwiZyIsMl1d
      \[\begin{tikzcd}
        && A \\
        C &&&& {} \\
        && B
        \arrow["f", from=2-1, to=1-3]
        \arrow["g"', from=2-1, to=3-3]
      \end{tikzcd}\]
      Applying $\points$ takes us to $\slos$, at which point we can use the AP of strict linear
      orders to get the commuting square
      % https://q.uiver.app/?q=WzAsNCxbMiwwLCJcXHBvaW50c1tBXSJdLFsyLDIsIlxccG9pbnRzW0JdIl0sWzAsMSwiXFxwb2ludHNbQ10iXSxbNCwxLCJcXE9tZWdhIl0sWzIsMCwiXFxwb2ludHNbZl0iXSxbMiwxLCJcXHBvaW50c1tnXSIsMl0sWzAsMywiZiciXSxbMSwzLCJnJyIsMl1d
      \[\begin{tikzcd}
        && {\points[A]} \\
        {\points[C]} &&&& \Omega \\
        && {\points[B]}
        \arrow["{\points[f]}", from=2-1, to=1-3]
        \arrow["{\points[g]}"', from=2-1, to=3-3]
        \arrow["{f'}", from=1-3, to=2-5]
        \arrow["{g'}"', from=3-3, to=2-5]
      \end{tikzcd}\]
      Now going back to $\aias$ gives the commuting diagram
      % https://q.uiver.app/?q=WzAsOSxbNCwwLCJcXGludGVyW1xccG9pbnRzW0FdXSJdLFs0LDIsIlxcaW50ZXJbXFxwb2ludHNbQl1dIl0sWzIsMSwiXFxpbnRlcltcXHBvaW50c1tDXV0iXSxbNiwxLCJcXGludGVyW1xcT21lZ2FdIl0sWzAsMSwiQyJdLFswLDBdLFsxLDBdLFsyLDAsIkEiXSxbMiwyLCJCIl0sWzIsMCwiXFxpbnRlcltcXHBvaW50c1tmXV0iLDFdLFsyLDEsIlxcaW50ZXJbXFxwb2ludHNbZ11dIiwxXSxbMCwzLCJcXGludGVyW2YnXSJdLFsxLDMsIlxcaW50ZXJbZyddIiwyXSxbOCwxLCJcXHVuaXR7Qn0iLDJdLFs3LDAsIlxcdW5pdHtBfSJdLFs0LDIsIlxcdW5pdHtDfSIsMV0sWzQsNywiZiJdLFs0LDgsImciLDJdXQ==
      \[\begin{tikzcd}
        {} & {} & A && {\inter[\points[A]]} \\
        C && {\inter[\points[C]]} &&&& {\inter[\Omega]} \\
        && B && {\inter[\points[B]]}
        \arrow["{\inter[\points[f]]}"{description}, from=2-3, to=1-5]
        \arrow["{\inter[\points[g]]}"{description}, from=2-3, to=3-5]
        \arrow["{\inter[f']}", from=1-5, to=2-7]
        \arrow["{\inter[g']}"', from=3-5, to=2-7]
        \arrow["{\unit{B}}"', from=3-3, to=3-5]
        \arrow["{\unit{A}}", from=1-3, to=1-5]
        \arrow["{\unit{C}}"{description}, from=2-1, to=2-3]
        \arrow["f", from=2-1, to=1-3]
        \arrow["g"', from=2-1, to=3-3]
      \end{tikzcd}\]
      For the AP of the finite interval algebras, we are only interested in the outer square. The
      necessary maps are then $\inter[f'] \circ \unit{A}$ and $\inter[g'] \circ \unit{B}$, both
      mapping into $\inter[\Omega]$.
    \item \textbf{EC}: Follows from the fact that $\laia$ is finite, similarly to the strict linear
      order case.
\end{itemize}
\end{proof}

\begin{thm}
  The Fraïssé limit of $\finaia$ is $\inter[\Q]$
\end{thm}
\begin{proof}
  To show that the Fraïssé limit of the finite interval algebras is $\inter[\Q]$, it suffices to
  show the following:
  \begin{itemize}
    \item $\age{\Q} = \finaia$:
    %TODO Show that the age of $\inter[\Q]$ is the finite interval algebras.
    \item $\inter[\Q]$ is homogeneous: fix two $\laia$-substructures $A,B \subseteq \inter[\Q]$,
      along with some $\laia$-isomorphism $f : A \to B$. In essence we have the following
      diagram in $\slos$:
      % https://q.uiver.app/?q=WzAsNCxbMCwyLCJBIl0sWzIsMiwiQiJdLFswLDAsIlxcaW50ZXJbUV0iXSxbMiwwLCJcXGludGVyW1FdIl0sWzAsMSwiZiJdLFswLDIsImkiXSxbMSwzLCJqIiwyXV0=
      \[\begin{tikzcd}
        {\inter[Q]} && {\inter[Q]} \\
        \\
        A && B
        \arrow["f", from=3-1, to=3-3]
        \arrow["i", from=3-1, to=1-1]
        \arrow["j"', from=3-3, to=1-3]
      \end{tikzcd}\]
      Applying $\inter$ to move to linear orders, we can postcompose $\points[i]$ and $\points[j]$ with
      the counit at $\Q$ to realise $\points[A]$ and $\points[B]$ as $\lslo$-substructures of $\Q$.
      Now, $\points[f]$ is still an isomorphism as these are preserved by functors, and using the
      fact that $\Q$ is homogeneous, we can extend $\points[f]$ to an isomorphism $g$, giving the
      commuting diagram
      % https://q.uiver.app/?q=WzAsNyxbMCw0LCJcXHBvaW50c1tBXSJdLFsyLDQsIlxccG9pbnRzW0JdIl0sWzAsMiwiXFxwb2ludHNbXFxpbnRlcltcXFFdXSJdLFsyLDIsIlxccG9pbnRzW1xcaW50ZXJbXFxRXV0iXSxbMCwwLCJcXFEiXSxbMiwwLCJcXFEiXSxbMCwxXSxbMCwxLCJcXHBvaW50c1tmXSJdLFswLDIsIlxccG9pbnRzW2ldIl0sWzIsNCwiXFxjb3VuaXR7XFxRfSJdLFszLDUsIlxcY291bml0e1xcUX0iLDJdLFsxLDMsIlxccG9pbnRzW2pdIiwyXSxbNCw1LCJnIl1d
      \[\begin{tikzcd}
        \Q && \Q \\
        {} \\
        {\points[\inter[\Q]]} && {\points[\inter[\Q]]} \\
        \\
        {\points[A]} && {\points[B]}
        \arrow["{\points[f]}", from=5-1, to=5-3]
        \arrow["{\points[i]}", from=5-1, to=3-1]
        \arrow["{\counit{\Q}}", from=3-1, to=1-1]
        \arrow["{\counit{\Q}}"', from=3-3, to=1-3]
        \arrow["{\points[j]}"', from=5-3, to=3-3]
        \arrow["g", from=1-1, to=1-3]
      \end{tikzcd}\]
      Finally, we apply $\inter$ to bring us back to interval algebras, where we have the following
      diagram:
      % https://q.uiver.app/?q=WzAsNixbMCw0LCJBIl0sWzMsNCwiQiJdLFswLDIsIlxcaW50ZXJbXFxwb2ludHNbQV1dIl0sWzMsMiwiXFxpbnRlcltcXHBvaW50c1tCXV0iXSxbMCwwLCJcXGludGVyW1xcUV0iXSxbMywwLCJcXGludGVyW1xcUV0iXSxbMCwxLCJmIl0sWzAsMiwiXFx1bml0e0F9Il0sWzEsMywiXFx1bml0e0J9IiwyXSxbMiwzLCJcXGludGVyW1xccG9pbnRzW2ZdXSJdLFszLDUsIlxcaW50ZXJbXFxjb3VuaXR7UX1cXGNpcmNcXHBvaW50c1tqXV0iXSxbMiw0LCJcXGludGVyW1xcY291bml0e1F9XFxjaXJjXFxwb2ludHNbaV1dIiwyXSxbNCw1LCJcXGludGVyW2ddIl0sWzAsNCwiaSIsMCx7ImN1cnZlIjotNX1dLFsxLDUsImoiLDIseyJjdXJ2ZSI6NX1dXQ==
      \[\begin{tikzcd}
        {\inter[\Q]} &&& {\inter[\Q]} \\
        \\
        {\inter[\points[A]]} &&& {\inter[\points[B]]} \\
        \\
        A &&& B
        \arrow["f", from=5-1, to=5-4]
        \arrow["{\unit{A}}", from=5-1, to=3-1]
        \arrow["{\unit{B}}"', from=5-4, to=3-4]
        \arrow["{\inter[\points[f]]}", from=3-1, to=3-4]
        \arrow["{\inter[\counit{Q}\circ\points[j]]}", from=3-4, to=1-4]
        \arrow["{\inter[\counit{Q}\circ\points[i]]}"', from=3-1, to=1-1]
        \arrow["{\inter[g]}", from=1-1, to=1-4]
        \arrow["i", curve={height=-60pt}, from=5-1, to=1-1]
        \arrow["j"', curve={height=60pt}, from=5-4, to=1-4]
      \end{tikzcd}\]
      The top square must commute, since it commuted for linear orders,
      and the bottom square must commute since the unit is a natural transformation.
      %TODO Check that i and j commute with the relevant things
  \end{itemize}
\end{proof}


% ------ Stability theory

\begin{thm}
  An interval algebra $A$ is stable if and only if it is finite.
\end{thm}
\begin{proof}
  Finite models are always stable since they cannot have infinitely many elements.

  For an infinite model $A$, we can linearly order the intervals in $A$, showing $A$ is unstable.
\end{proof}

\begin{thm}
  If a model M is interpretable in an NIP model, then M must also be NIP
\end{thm}
\begin{proof}
  Prove the contrapositive (if M is IP, then anything M is interpretable in is IP)
  If M is interpretable in $N^k$, and there is a formula $\phi$ which has the IP in M,
  then viewing this formula as a formula over $N$ implies that $N$ is also IP.
\end{proof}

\begin{cor}
  Given an NIP linear order $L$, the interval algebra $\inter[L]$ is also NIP.
\end{cor}
\begin{proof}
  $\inter[L]$ is interpretable in $L$ by its construction.
\end{proof}

\begin{thm}
  Given an linear order $L$ with no upper bound (or no lower bound), $L$ is an NIP model if and only
  if $\inter[L]$ is an NIP model.
\end{thm}
\begin{proof}
  The left to right implication holds by the previous corollary.
  For the right to left implication, we notice that we can interpret the linear order $L$ in
  $\inter[L]$ as it has no upper bound (use a construction similar to Points where we only consider
  the end points of an interval -- the lower bound analogue follows from the start point
  construction).
\end{proof}

\end{document}
