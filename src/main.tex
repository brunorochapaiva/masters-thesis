\documentclass[11pt % Font size
              ]{article}
\usepackage[ a4paper % Paper size and format
           , onecolumn % Number of columns: onecolumn or twocolumn
           , total={6in, 8in} % Text area: {width, height}
           ]{geometry}

\title{A Model Theoretic Study of Allen's Interval Algebra}
\author{Bruno da Rocha Paiva - bmd18}
\date{} % Date of writing: empty, today or 25.12.00

\usepackage{amsmath}
\usepackage{amsthm}
\usepackage{amssymb}
\usepackage{parskip}
\usepackage{tikz}
\usepackage{quiver}
\usepackage{xcolor}


\usepackage{subfiles}

% Used to scale tikz-cd diagrams properly
\tikzcdset{scale cd/.style={every label/.append style={scale=#1},
    cells={nodes={scale=#1}}}}

% General Symbols
\newcommand{\N}{\mathbb{N}}
\newcommand{\Z}{\mathbb{Z}}
\newcommand{\Q}{\mathbb{Q}}
\newcommand{\R}{\mathbb{R}}
\newcommand{\congto}{\xrightarrow{\sim}}
\newcommand{\letin}[2]{\text{let }{#1}\text{ in }{#2}}

% Category Theory
\newcommand{\id}[1]{\text{id}_{#1}}
\newcommand{\unit}[1]{\eta_{#1}}
\newcommand{\counit}[1]{\epsilon_{#1}}

% Model Theory
\newcommand{\lang}{\mathcal{L}}
\newcommand{\theory}{\mathbb{T}}
\newcommand{\age}[1]{\text{Age}\left(#1\right)}

\newcommand{\finslo}{\textbf{FCh}}
\newcommand{\finaia}{\textbf{FIA}}


\newcommand{\lslo}{\lang_\text{SLO}}
\newcommand{\tslo}{\theory_\text{SLO}}

% Allen Interval Algebra Symbols
\newcommand{\laia}{\lang_\text{AIA}}
\newcommand{\taia}{\theory_\text{AIA}}
\newcommand{\istart}[1]{#1_{-}}
\newcommand{\iend}[1]{#1_{+}}
\newcommand{\aiaindex}{\{<,m,o,s,f,d\}}

% Allen Interval Algebra Relations
\newcommand{\aiaarrow}[1]{\overset{#1}{\longrightarrow}}
\newcommand{\raiaarrow}[1]{\overset{#1}{\longleftarrow}}
\newcommand{\before}{\aiaarrow{<}}
\newcommand{\meets}{\aiaarrow{m}}
\newcommand{\overlaps}{\aiaarrow{o}}
\newcommand{\starts}{\aiaarrow{s}}
\newcommand{\finishes}{\aiaarrow{f}}
\newcommand{\contained}{\aiaarrow{d}}
\newcommand{\after}{\raiaarrow{<}}
\newcommand{\metby}{\raiaarrow{m}}
\newcommand{\overlappedby}{\raiaarrow{o}}
\newcommand{\startedby}{\raiaarrow{s}}
\newcommand{\finishedby}{\raiaarrow{f}}
\newcommand{\contains}{\raiaarrow{d}}

% Categories of models
\newcommand{\mods}[2]{\text{Mod}\left(#1,#2\right)}
\newcommand{\aias}{\textbf{AIA}}
\newcommand{\slos}{\textbf{SLO}}

% Intervals construction
\newcommand{\inter}[1][-]{\text{Int}\left(#1\right)}
\newcommand{\defformula}[1]{\phi_{#1}}
\newcommand{\beforef}{\defformula{\before}}
\newcommand{\meetsf}{\defformula{\meets}}
\newcommand{\overlapsf}{\defformula{\overlaps}}
\newcommand{\startsf}{\defformula{\starts}}
\newcommand{\finishesf}{\defformula{\finishes}}
\newcommand{\containedf}{\defformula{\contained}}

\newcommand{\lexord}{\defformula{\text{lex}}}

% Points construction
\newcommand{\psim}{\sim}
\newcommand{\peq}{\phi_\sim}
\newcommand{\plt}{\phi_<}
\newcommand{\points}[1][-]{\text{Pts}\left(#1\right)}

% Environments
\theoremstyle{plain}
\newtheorem{thm}{Theorem}%[subsection]
\newtheorem{lemma}[thm]{Lemma}
\newtheorem{prop}[thm]{Proposition}
\newtheorem{cor}[thm]{Corollary}

\theoremstyle{definition}
\newtheorem{defn}[thm]{Definition}

\theoremstyle{remark}
\newtheorem*{rem}{Remark}

\newcommand{\transreltable}[3]{
  \begin{table}[ht]
    \centering
    \begin{tabular}{| c | c | c || c | c | c |}
      \hline
      $I \longrightarrow J$ & $J \longrightarrow K$ & $I \longrightarrow K$ &
        $I \longrightarrow J$ & $J \longrightarrow K$ & $I \longrightarrow K$ \\
      \hline\hline
      #3
      \hline
    \end{tabular}
    \caption{#2}
    \label{tab:#1}
  \end{table}
}

\newcommand{\llrow}{$<$ & $<$ & $<$}
\newcommand{\lmrow}{$<$ & $m$ & $<$}
\newcommand{\lorow}{$<$ & $o$ & $<$}
\newcommand{\lsrow}{$<$ & $s$ & $<$}
\newcommand{\lfrow}{$<$ & $f$ & $<mosd$}
\newcommand{\ldrow}{$<$ & $d$ & $<mosd$}
\newcommand{\lerow}{$<$ & $=$ & $<$}
\newcommand{\lLrow}{$<$ & $>$ & full}
\newcommand{\lMrow}{$<$ & $M$ & $<mosd$}
\newcommand{\lOrow}{$<$ & $O$ & $<mosd$}
\newcommand{\lSrow}{$<$ & $S$ & $<$}
\newcommand{\lFrow}{$<$ & $F$ & $<$}
\newcommand{\lDrow}{$<$ & $D$ & $<$}

\newcommand{\mlrow}{$m$ & $<$ & $<$}
\newcommand{\mmrow}{$m$ & $m$ & $<$}
\newcommand{\morow}{$m$ & $o$ & $<$}
\newcommand{\msrow}{$m$ & $s$ & $m$}
\newcommand{\mfrow}{$m$ & $f$ & $osd$}
\newcommand{\mdrow}{$m$ & $d$ & $osd$}
\newcommand{\merow}{$m$ & $=$ & $m$}
\newcommand{\mLrow}{$m$ & $>$ & $>MOSD$}
\newcommand{\mMrow}{$m$ & $M$ & $f=F$}
\newcommand{\mOrow}{$m$ & $O$ & $osd$}
\newcommand{\mSrow}{$m$ & $S$ & $m$}
\newcommand{\mFrow}{$m$ & $F$ & $<$}
\newcommand{\mDrow}{$m$ & $D$ & $<$}

\newcommand{\olrow}{$o$ & $<$ & $<$}
\newcommand{\omrow}{$o$ & $m$ & $<$}
\newcommand{\oorow}{$o$ & $o$ & $<mo$}
\newcommand{\osrow}{$o$ & $s$ & $o$}
\newcommand{\ofrow}{$o$ & $f$ & $osd$}
\newcommand{\odrow}{$o$ & $d$ & $osd$}
\newcommand{\oerow}{$o$ & $=$ & $o$}
\newcommand{\oLrow}{$o$ & $>$ & $>MOSD$}
\newcommand{\oMrow}{$o$ & $M$ & $OSD$}
\newcommand{\oOrow}{$o$ & $O$ & concur}
\newcommand{\oSrow}{$o$ & $S$ & $oFD$}
\newcommand{\oFrow}{$o$ & $F$ & $<mo$}
\newcommand{\oDrow}{$o$ & $D$ & $<moFD$}

\newcommand{\slrow}{$s$ & $<$ & $<$}
\newcommand{\smrow}{$s$ & $m$ & $<$}
\newcommand{\sorow}{$s$ & $o$ & $<mo$}
\newcommand{\ssrow}{$s$ & $s$ & $s$}
\newcommand{\sfrow}{$s$ & $f$ & $d$}
\newcommand{\sdrow}{$s$ & $d$ & $d$}
\newcommand{\serow}{$s$ & $=$ & $s$}
\newcommand{\sLrow}{$s$ & $>$ & $>$}
\newcommand{\sMrow}{$s$ & $M$ & $M$}
\newcommand{\sOrow}{$s$ & $O$ & $fdO$}
\newcommand{\sSrow}{$s$ & $S$ & $s=S$}
\newcommand{\sFrow}{$s$ & $F$ & $<mo$}
\newcommand{\sDrow}{$s$ & $D$ & $<moFD$}

\newcommand{\flrow}{$f$ & $<$ & $<$}
\newcommand{\fmrow}{$f$ & $m$ & $m$}
\newcommand{\forow}{$f$ & $o$ & $osd$}
\newcommand{\fsrow}{$f$ & $s$ & $d$}
\newcommand{\ffrow}{$f$ & $f$ & $f$}
\newcommand{\fdrow}{$f$ & $d$ & $d$}
\newcommand{\ferow}{$f$ & $=$ & $f$}
\newcommand{\fLrow}{$f$ & $>$ & $>$}
\newcommand{\fMrow}{$f$ & $M$ & $>$}
\newcommand{\fOrow}{$f$ & $O$ & $>MO$}
\newcommand{\fSrow}{$f$ & $S$ & $>MO$}
\newcommand{\fFrow}{$f$ & $F$ & $f=F$}
\newcommand{\fDrow}{$f$ & $D$ & $>MOSD$}

\newcommand{\dlrow}{$d$ & $<$ & $<$}
\newcommand{\dmrow}{$d$ & $m$ & $<$}
\newcommand{\dorow}{$d$ & $o$ & $<mosd$}
\newcommand{\dsrow}{$d$ & $s$ & $d$}
\newcommand{\dfrow}{$d$ & $f$ & $d$}
\newcommand{\ddrow}{$d$ & $d$ & $d$}
\newcommand{\derow}{$d$ & $=$ & $d$}
\newcommand{\dLrow}{$d$ & $>$ & $>$}
\newcommand{\dMrow}{$d$ & $M$ & $>$}
\newcommand{\dOrow}{$d$ & $O$ & $df>OM$}
\newcommand{\dSrow}{$d$ & $S$ & $df>OM$}
\newcommand{\dFrow}{$d$ & $F$ & $<mosd$}
\newcommand{\dDrow}{$d$ & $D$ & full}

\newcommand{\elrow}{$=$ & $<$ & $<$}
\newcommand{\emrow}{$=$ & $m$ & $m$}
\newcommand{\eorow}{$=$ & $o$ & $o$}
\newcommand{\esrow}{$=$ & $s$ & $s$}
\newcommand{\efrow}{$=$ & $f$ & $f$}
\newcommand{\edrow}{$=$ & $d$ & $d$}
\newcommand{\eerow}{$=$ & $=$ & $=$}
\newcommand{\eLrow}{$=$ & $>$ & $>$}
\newcommand{\eMrow}{$=$ & $M$ & $M$}
\newcommand{\eOrow}{$=$ & $O$ & $O$}
\newcommand{\eSrow}{$=$ & $S$ & $S$}
\newcommand{\eFrow}{$=$ & $F$ & $F$}
\newcommand{\eDrow}{$=$ & $D$ & $D$}

\newcommand{\Llrow}{$>$ & $<$ & full}
\newcommand{\Lmrow}{$>$ & $m$ & $df>OM$}
\newcommand{\Lorow}{$>$ & $o$ & $df>OM$}
\newcommand{\Lsrow}{$>$ & $s$ & $df>OM$}
\newcommand{\Lfrow}{$>$ & $f$ & $>$}
\newcommand{\Ldrow}{$>$ & $d$ & $df>OM$}
\newcommand{\Lerow}{$>$ & $=$ & $>$}
\newcommand{\LLrow}{$>$ & $>$ & $>$}
\newcommand{\LMrow}{$>$ & $M$ & $>$}
\newcommand{\LOrow}{$>$ & $O$ & $>$}
\newcommand{\LSrow}{$>$ & $S$ & $>$}
\newcommand{\LFrow}{$>$ & $F$ & $>$}
\newcommand{\LDrow}{$>$ & $D$ & $>$}

\newcommand{\Mlrow}{$M$ & $<$ & $<moFD$}
\newcommand{\Mmrow}{$M$ & $m$ & $s=S$}
\newcommand{\Morow}{$M$ & $o$ & $fdO$}
\newcommand{\Msrow}{$M$ & $s$ & $fdO$}
\newcommand{\Mfrow}{$M$ & $f$ & $M$}
\newcommand{\Mdrow}{$M$ & $d$ & $fdO$}
\newcommand{\Merow}{$M$ & $=$ & $M$}
\newcommand{\MLrow}{$M$ & $>$ & $>$}
\newcommand{\MMrow}{$M$ & $M$ & $>$}
\newcommand{\MOrow}{$M$ & $O$ & $>$}
\newcommand{\MSrow}{$M$ & $S$ & $>$}
\newcommand{\MFrow}{$M$ & $F$ & $M$}
\newcommand{\MDrow}{$M$ & $D$ & $>$}

\newcommand{\Olrow}{$O$ & $<$ & $<moFD$}
\newcommand{\Omrow}{$O$ & $m$ & $oFD$}
\newcommand{\Oorow}{$O$ & $o$ & concur}
\newcommand{\Osrow}{$O$ & $s$ & $fdO$}
\newcommand{\Ofrow}{$O$ & $f$ & $O$}
\newcommand{\Odrow}{$O$ & $d$ & $fdO$}
\newcommand{\Oerow}{$O$ & $=$ & $O$}
\newcommand{\OLrow}{$O$ & $>$ & $>$}
\newcommand{\OMrow}{$O$ & $M$ & $>$}
\newcommand{\OOrow}{$O$ & $O$ & $>MO$}
\newcommand{\OSrow}{$O$ & $S$ & $>MO$}
\newcommand{\OFrow}{$O$ & $F$ & $OSD$}
\newcommand{\ODrow}{$O$ & $D$ & $>MOSD$}

\newcommand{\Slrow}{$S$ & $<$ & $<moFD$}
\newcommand{\Smrow}{$S$ & $m$ & $oFD$}
\newcommand{\Sorow}{$S$ & $o$ & $oFD$}
\newcommand{\Ssrow}{$S$ & $s$ & $s=S$}
\newcommand{\Sfrow}{$S$ & $f$ & $O$}
\newcommand{\Sdrow}{$S$ & $d$ & $fdO$}
\newcommand{\Serow}{$S$ & $=$ & $S$}
\newcommand{\SLrow}{$S$ & $>$ & $>$}
\newcommand{\SMrow}{$S$ & $M$ & $M$}
\newcommand{\SOrow}{$S$ & $O$ & $O$}
\newcommand{\SSrow}{$S$ & $S$ & $S$}
\newcommand{\SFrow}{$S$ & $F$ & $D$}
\newcommand{\SDrow}{$S$ & $D$ & $D$}

\newcommand{\Flrow}{$F$ & $<$ & $<$}
\newcommand{\Fmrow}{$F$ & $m$ & $m$}
\newcommand{\Forow}{$F$ & $o$ & $o$}
\newcommand{\Fsrow}{$F$ & $s$ & $o$}
\newcommand{\Ffrow}{$F$ & $f$ & $f=F$}
\newcommand{\Fdrow}{$F$ & $d$ & $osd$}
\newcommand{\Ferow}{$F$ & $=$ & $F$}
\newcommand{\FLrow}{$F$ & $>$ & $>MOSD$}
\newcommand{\FMrow}{$F$ & $M$ & $OSD$}
\newcommand{\FOrow}{$F$ & $O$ & $OSD$}
\newcommand{\FSrow}{$F$ & $S$ & $D$}
\newcommand{\FFrow}{$F$ & $F$ & $F$}
\newcommand{\FDrow}{$F$ & $D$ & $D$}

\newcommand{\Dlrow}{$D$ & $<$ & $<moFD$}
\newcommand{\Dmrow}{$D$ & $m$ & $oFD$}
\newcommand{\Dorow}{$D$ & $o$ & $oFD$}
\newcommand{\Dsrow}{$D$ & $s$ & $oFD$}
\newcommand{\Dfrow}{$D$ & $f$ & $OSD$}
\newcommand{\Ddrow}{$D$ & $d$ & concur}
\newcommand{\Derow}{$D$ & $=$ & $D$}
\newcommand{\DLrow}{$D$ & $>$ & $>MOSD$}
\newcommand{\DMrow}{$D$ & $M$ & $OSD$}
\newcommand{\DOrow}{$D$ & $O$ & $OSD$}
\newcommand{\DSrow}{$D$ & $S$ & $D$}
\newcommand{\DFrow}{$D$ & $F$ & $D$}
\newcommand{\DDrow}{$D$ & $D$ & $D$}

\begin{document}

\maketitle

\section{Axiomatisation of Interval Algebras}

\begin{defn}
  We define the language of strict linear orders $\lslo$ as the single binary relation $\{ < \}$.

  We define the theory of strict linear orders as
  \begin{align*}
    \tslo = \{ & \forall a, \lnot\ (a < a), \\
              & \forall a, \forall b, \forall c, (a < b) \land (b < c) \rightarrow (a < c) \\
              & \forall a, \forall b, (a < b) \lor (a = b) \lor (b < a) \}
  \end{align*}
\end{defn}

\begin{defn}
  We define the language of Allen interval algebras $\laia$ as
  \begin{equation*}
    \laia = \{ \before, \meets, \overlaps, \starts, \finishes, \contained,
               \after, \metby, \overlappedby, \startedby, \finishedby, \contains \}
  \end{equation*}

  {\color{orange} TODO write out the axioms of interval algebras}
\end{defn}

{\color{orange} TODO show it suffices to consider relations of the form -> when showing a function is
an embedding}
% mention how we only need to prove the forward implication to show something is a
% $\laia$-embedding. This is enough since the relations are mutually-exclusive and exhaustive.
% Also it is enough to check that the relations of the form -> are preserved -- if that is the case
% then the relations of the form <- are automatically also preserved by "duality"?/"symmetry"?

% ------ Justifying this axiomatisation of interval algebras

\begin{defn}
  Given a linear order $L$ we define its set of non-zero intervals $\inter[P]$ as the set
  \begin{equation*}
    \inter[L] = \left\{(x_1,x_2)\ |\ x_1 < x_2 \right\} \subseteq L^2
  \end{equation*}
  We can turn this into a $\laia$-structure under the interpretations:
  \begin{itemize}
    \item $(x_1,x_2) \aiaarrow{i} (y_1,y_2)$ if and only if
      $L \models \defformula{\aiaarrow{i}}(x_1,x_2,y_1,y_2)$.
    \item $(x_1,x_2) \raiaarrow{i} (y_1,y_2)$ if and only if
      $L \models \defformula{\aiaarrow{i}}(y_1,y_2,x_1,x_2)$.
  \end{itemize}
  where we range $i$ over the indexing set $\aiaindex$ and define the first order
  $\lslo$-formulas by
  \begin{align*}
    \beforef(x_1,x_2,y_1,y_2)    & = \left(x_1 < x_2\right) \land \left(x_2 < y_1\right) \land
      \left(y_1 < y_2\right) \\
    \meetsf(x_1,x_2,y_1,y_2)     & = \left(x_1 < x_2\right) \land \left(x_2 = y_1\right) \land
      \left(y_1 < y_2\right) \\
    \overlapsf(x_1,x_2,y_1,y_2)  & = \left(x_1 < y_1\right) \land \left(y_1 < x_2\right) \land
      \left(x_2 < y_2\right) \\
    \startsf(x_1,x_2,y_1,y_2)    & = \left(x_1 = y_1\right) \land \left(y_1 < x_2\right) \land
      \left(x_2 < y_2\right) \\
    \finishesf(x_1,x_2,y_1,y_2)  & = \left(y_1 < x_1\right) \land \left(x_1 < x_2\right) \land
      \left(x_2 = y_2\right) \\
    \containedf(x_1,x_2,y_1,y_2) & = \left(y_1 < x_1\right) \land \left(x_1 < x_2\right) \land
      \left(x_2 < y_2\right) \\
  \end{align*}
\end{defn}

\begin{thm}
  Given a strict linear order $L$, $\inter[L]$ is a model of $\taia$ under the above
  interpretations.
\end{thm}
\begin{proof}
  {\color{orange} TODO Tedious proof by cases goes here}
\end{proof}

\begin{cor}
  Allen's interval algebras are satisfiable
\end{cor}

The theory is satisfiable in a reasonable way, including the expected models induced from linear
orders. To further support this tight connection, we see interval algebras also induce a quite
reasonable linear order

\begin{defn}
  Given an Allen interval algebra $A$, we define $\points[A] = \frac{A + A}{\psim}$ where
  $\psim$ is an equivalence relation on the set $A + A$ defined by
  \begin{equation*}
    (n,I) \psim (m,J) \iff A \models \peq(n,m)(I,J)
  \end{equation*}
  Where $\peq$ assigns $\laia$-formulas with open variables $I,J$ to pairs $(n,m) \in \{0,1\}^2$
  \begin{equation*}
    \peq(n,m)(I,J) = \begin{cases}
      I \starts J \lor I \startedby J \lor I = J \quad & \text{if } n = m = 0\\
      I \finishes J \lor I \finishedby J \lor I = J    & \text{if } n = m = 1 \\
      I \metby J                                       & \text{if } n = 0,\ m = 1 \\
      I \meets J                                       & \text{if } n = 1,\ m = 0
    \end{cases}
  \end{equation*}
\end{defn}

\begin{thm}
  Given an Allen interval algebra $A$, the interpretation of the symbol $<$ in $\points[A]$ given by
  $(n,I) < (m,J) \iff A \models \plt(n,m)(I,J)$ is well-defined and turns $A$ into a model of
  $\tslo$.

  Here, $\plt$ is a function assigning $\laia$-formulas with open variables $I,J$ to pairs
  $(n,m) \in \{0,1\}^2$ defined as:
  \begin{equation*}
    \plt(n,m)(I,J) = \begin{cases}
      (I \before J) \lor (I \meets J) \lor (I \overlaps J)  \lor (I \finishedby J)
                    \lor (I \contains J)        & \text{if } n = m = 0\\
      (I \before J) \lor (I \meets J) \lor (I \overlaps J) \lor (I \starts J)
                  \lor (I \contained J)         & \text{if } n = m = 1 \\
      \lnot\,(I \after J) \land \lnot\,(I \metby J) \land \lnot\,(I \startedby J)
                                                & \text{if } n = 0,\ m = 1 \\
      I \before J                               & \text{if } n = 1,\ m = 0
    \end{cases}
  \end{equation*}
\end{thm}
\begin{proof}
  {\color{orange} TODO Show that $<$ is well defined}

  Now we know that our definition of $<$ does not depend on a choice of representative, we need
  to check whether it satisfies $\tslo$:
  \begin{itemize}
    \item \textbf{irreflexivity}: Fix some element $[(n,I)] \in \points[A]$. Since the interval
      algebra relations (along with equality) are mutually exclusive and $I = I$, no other relation
      can hold for the pair $(I,I)$. Regardless of the value of $n$, $\plt(n,n)(I,I)$ does not
      include $I = I$ as a disjunct, so $A \not\models \plt(n,n)(I,I)$ and $<$ is irreflexive.
    \item \textbf{transitivity}: {\color{orange} TODO Show transitivity}
    \item \textbf{trichotomy}: {\color{orange} TODO Show trichotomy}
  \end{itemize}
\end{proof}

{\color{blue} TODO Consider adding table justifying these definitions? Something like:

| I relation J | drawing of I and J | is I- less than J- | is I+ lt J+ | is I- lt J + | is I+ lt J - |}

Now that this axiomatisation is justified, show that we can't remove any axioms without making
it into something we do not want.

\newpage
\section{A Detour Into Category Theory}

\begin{defn}
  Given a theory $\theory$ over language $\lang$, we denote by $\mods{\lang}{\theory}$ the category
  with objects the models of $\theory$ and arrows the $\lang$-embeddings.
\end{defn}

\begin{rem}
  For brevity, we introduce the notation:
  \begin{equation*}
    \slos := \mods{\lslo}{\tslo} \text{ and } \aias := \mods{\laia}{\taia}
  \end{equation*}
\end{rem}

\begin{thm}
  We can turn $\inter$ into a functor
  \begin{equation*}
    \inter : \slos \to \aias
  \end{equation*}
  by sending arrows $f : M \to N$ in $\slos$ to  $\inter[f] : \inter[M] \to \inter[N]$ defined by
  \begin{equation*}
    \inter[f](x_1, x_2) = (f(x_1), f(x_2))
  \end{equation*}
\end{thm}
\begin{proof}
  First we show that for a $\lslo$-embedding $f : L \to M$ between strict linear orders
  $L,M$, the mapping $\inter[f]$ gives a an $\laia$-embedding. Consider the relations in $\laia$,
  and their interpretations in $\inter[L]$: all of the relations are defined by quantifier-free
  $\lslo$-formulas, whose truth value must be preserved under $\lslo$ embeddings like $f$. Since
  $\inter[f]$ simply applies $f$ pointwise, $\inter[f]$ must preserve the truth value of the
  relation symbols in $\laia$, in other words, it is an $\laia$-embedding.

  Now we just need to check that $\inter$ satisfies the two functor axioms:
  \begin{itemize}
    \item \textbf{preserves identity arrows}: Fix some strict linear order $L$ and some interval
      $(x,y) \in \inter[L]$, then
      \begin{equation*}
        \inter[\id{L}](x,y) = (\id{L}(x), \id{L}(y)) = (x,y)
      \end{equation*}
    \item \textbf{respects arrow composition}: Fix three strict linear orders $L,M,N$ along with
      arrows $f : L \to M$, $g : M \to N$ and some interval $(x,y) \in \inter[L]$, then
      \begin{align*}
        \inter[g \circ f](x,y)
          & = (g \circ f(x), g \circ f(y)) \\
          & = (g(f(x)), g(f(y))) \\
          & = \inter[g](\inter[f](x,y)) \\
          & = \inter[g]\circ\inter[f](x,y)
      \end{align*}
  \end{itemize}
\end{proof}

\begin{thm}
  We can turn $\points$ into a functor
  \begin{equation*}
    \points : \aias \to \slos
  \end{equation*}
  by sending arrows $f : A \to B$ in $\aias$ to $\points[f] : \points[A] \to \points[B]$
  defined by
  \begin{equation*}
    \points[f](0, I) = (0, f(I)) \qquad\text{and}\qquad \points[f](1,I) = (1, f(I))
  \end{equation*}
\end{thm}
\begin{proof}
  Given an arrow $f : A \to B$, we must check that $\points[f]$ is a well defined map, and that it
  is an $\lslo$-embedding. These facts both follow by noticing that $f$ is an
  $\laia$-embedding, so it preserves the truth of quantifier-free $\laia$-formulas, so
  \begin{align*}
    (n,I) \psim (m,J)
      & \iff A \models \peq(n,m)(I,J) \\
      & \iff A \models \peq(n,m)(f(I),f(J)) \\
      & \iff (n,f(I)) \psim (m,f(J))
  \end{align*}
  and similarly
  \begin{align*}
    [(n,I)] < [(m,J)]
      & \iff A \models \plt(n,m)(I,J) \\
      & \iff A \models \plt(n,m)(f(I),f(J)) \\
      & \iff \points[f]([(n,I)]) < \points[f]([(m,J)])
  \end{align*}
  for any two $(n,I),(m,J) \in A + A$, since $\peq(n,m)$ and $\plt(n,m)$ are always quantifier-free.

  Next, to see that $\points$ satisfies the functor axioms:
  \begin{itemize}
    \item \textbf{preserves identity arrows}: Fix some interval algebra $A$ and some element
    $[(n, I)] \in \points[A]$. Then notice that
      \begin{equation*}
        \points[\id{A}]([(n,I)]) = [(n, \id{A}(I))] = [(n,I)] = \id{\points[A]}([(n,I)])
      \end{equation*}
      Hence $\points[\id{A}] = \id{\points[A]}$.
    \item \textbf{respects arrow composition}: Fix interval algebras $A,B,C$ along with arrows
      $f : A \to B$, $g : B \to C$. Then for all elements $[(n,I)] \in \points[A]$:
      \begin{align*}
        \points[g \circ f]([(n,I)])
          & = [(n,g \circ f (I))] \\
          & = [(n, g(f(I)))] \\
          & = \points[g]\left( \points[f]([n,I]) \right) \\
          & = \points[g] \circ \points[f] ([n,I])
      \end{align*}
      Hence $\points[g \circ f] = \points[g] \circ \points[f]$ as expected.
  \end{itemize}
\end{proof}

\begin{rem}
  From now on, given an interval algebra $A$ and interval $I \in A$, we will use
  $\istart{I} := [(0,I)]$ and $\iend{I} := [(1,I)]$ to refer to the respective elements of
  $\points[A]$.
\end{rem}

\begin{thm}
  $\points$ is left adjoint to $\inter$.
\end{thm}
\begin{proof}
  We prove this through the Hom-Set definition of an adjunction, so we wish to find an isomorphism
  $\slos(\points[A],L) \cong \aias(A,\inter[L])$ which is natural in both $A$ and $L$.

  We start by defining the forward map
  \begin{equation*}
    \phi_{A,L} : \slos(\points[A],L) \to \aias(A,\inter[L])
  \end{equation*}
  which sends an $\lslo$-embedding $f : \points[A] \to L$ to the $\laia$-embedding
  \begin{equation*}
    \phi_{A,L}(f) : A \to \inter[L] \quad\text{sending}\quad
      I \mapsto (f(\istart{I}), f(\iend{I}))
  \end{equation*}
  To see this is a $\laia$-embedding, recall that it suffices to consider the $\aiaarrow{i}$
  relations for $i \in \aiaindex$, so we fix two intervals $I,J \in A$ and then:
  \begin{equation*}
    \begin{split}
      I \before J
        & \iff   \iend{I}  <   \istart{J} \\
        & \iff f(\iend{I}) < f(\istart{J}) \\
        & \iff \phi_{A,L}(f)(I) \before \phi_{A,L}(f)(J) \\
      I \meets J
        & \iff   \iend{I}  =   \istart{J} \\
        & \iff f(\iend{I}) = f(\istart{J}) \\
        & \iff \phi_{A,L}(f)(I) \meets \phi_{A,L}(f)(J) \\
      I \overlaps J
        & \iff   \istart{I}  <   \istart{J}  <   \iend{I}  <   \iend{J} \\
        & \iff f(\istart{I}) < f(\istart{J}) < f(\iend{I}) < f(\iend{J}) \\
        & \iff \phi_{A,L}(f)(I) \overlaps \phi_{A,L}(f)(J)
      \end{split}
      \begin{split}
      I \starts J
        & \iff   \istart{I}  =   \istart{J}  <   \iend{I}  <   \iend{J} \\
        & \iff f(\istart{I}) = f(\istart{J}) < f(\iend{I}) < f(\iend{J}) \\
        & \iff \phi_{A,L}(f)(I) \starts \phi_{A,L}(f)(J) \\
      I \finishes J
        & \iff   \istart{J}  <   \istart{I}  <   \iend{I}  =   \iend{J} \\
        & \iff f(\istart{J}) < f(\istart{I}) < f(\iend{I}) = f(\iend{J}) \\
        & \iff \phi_{A,L}(f)(I) \finishes \phi_{A,L}(f)(J) \\
      I \contained J
        & \iff   \istart{J}  <   \istart{I}  <   \iend{I}  <   \iend{J} \\
        & \iff f(\istart{J}) < f(\istart{I}) < f(\iend{I}) < f(\iend{J}) \\
        & \iff \phi_{A,L}(f)(I) \contained \phi_{A,L}(f)(J)
    \end{split}
  \end{equation*}

  Next, we define the inverse map
  \begin{equation*}
    \psi_{A,L} : \aias(A,\inter[L]) \to \slos(\points[A],L)
  \end{equation*}
  which sends an $\laia$-embedding $g : A \to \inter[L]$ to the $\lslo$-embedding given by
  \begin{equation*}
    \psi_{A,L}(f) : \points[A] \to L \quad\text{sending}\quad (n, I) \mapsto
      \left(\letin{(x_0,x_1) := g(I)}{x_n}\right)
  \end{equation*}
  Now, we need to ensure these maps are well defined and that they are indeed $\lslo$-embeddings:
  {\color{orange}TODO Show this is well-defined and monotonic}

  We expect $\phi_{A,L}$ and $\psi_{A,L}$ to be inverses, which is confirmed by some quick
  computations:
  \begin{itemize}
    \item Pick  any $f : \points[A] \to L$ and $[(n,I)] \in \points[A]$, then
      \begin{align*}
        \psi_{A,L}\left(\phi_{A,L}(f)\right)([(n,I)])
          & = \left( \letin{(x_0,x_1) := \phi_{A,L}(f)(I)}{x_n} \right) \\
          & = \left( \letin{(x_0,x_1) := (f(\istart{I}), f(\iend{I}))}{x_n} \right) \\
          & = f([n, I])
      \end{align*}
    \item Pick any $g : A \to \inter[L]$ and $I \in A$, then
      \begin{equation*}
        \phi_{A,L}\left(\psi_{A,L}(g)\right)(I)
          = \left( \psi_{A,L}(g)(\istart{I}), \psi_{A,L}(g)(\iend{I}) \right)
          = g(I)
      \end{equation*}
  \end{itemize}

  Finally, we just have to check naturality of our isomorphism $\phi_{A,L}$. So pick a
  $\laia$-embedding $f : A \to B$ and a $\lslo$-embeddings $g : L \to M$, we need to show the
  following diagram commutes
  % If modifying diagram, don't forget to add [scale cd=0.99]
  % https://q.uiver.app/?q=WzAsNixbMywwLCJcXHNsb3MoXFxwb2ludHNbQV0sTCkiXSxbMywyLCJcXGFpYXMoQSxcXGludGVyW0xdKSJdLFswLDAsIlxcc2xvcyhcXHBvaW50c1tCXSxMKSJdLFswLDIsIlxcYWlhcyhCLFxcaW50ZXJbTF0pIl0sWzYsMiwiXFxhaWFzKEEsXFxpbnRlcltNXSkiXSxbNiwwLCJcXHNsb3MoXFxwb2ludHNbQV0sTSkiXSxbMiwwLCJcXHNsb3MoXFxwb2ludHNbZl0sTCkiXSxbMywxLCJcXGFpYXMoZixcXGludGVyW0xdKSIsMl0sWzIsMywiXFxwaGlfe0IsTH0iLDJdLFsxLDQsIlxcYWlhcyhBLFxcaW50ZXJbZ10pIiwyXSxbMCw1LCJcXHNsb3MoXFxwb2ludHNbQV0sZykiXSxbNSw0LCJcXHBoaV97QSxNfSJdLFswLDEsIlxccGhpX3tBLEx9IiwyXV0=
  \[\begin{tikzcd}[scale cd=0.99]
    {\slos(\points[B],L)} &&& {\slos(\points[A],L)} &&& {\slos(\points[A],M)} \\
    \\
    {\aias(B,\inter[L])} &&& {\aias(A,\inter[L])} &&& {\aias(A,\inter[M])}
    \arrow["{\slos(\points[f],L)}", from=1-1, to=1-4]
    \arrow["{\aias(f,\inter[L])}"', from=3-1, to=3-4]
    \arrow["{\phi_{B,L}}"', from=1-1, to=3-1]
    \arrow["{\aias(A,\inter[g])}"', from=3-4, to=3-7]
    \arrow["{\slos(\points[A],g)}", from=1-4, to=1-7]
    \arrow["{\phi_{A,M}}", from=1-7, to=3-7]
    \arrow["{\phi_{A,L}}"', from=1-4, to=3-4]
  \end{tikzcd}\]
  We do this by checking both squares individually:
  \begin{itemize}
    \item \textbf{left square}: Pick some $h : \slos(\points[B],L)$ and $I \in A$, then
      \begin{align*}
        \phi_{A,L}(\slos(\points[f],L)(h))(I)
          & = \phi_{A,L}(h \circ \points[f])(I) \\
          & = (h(\points[f](\istart{I})), h(\points[f](\iend{I}))) \\
          & = (h(\istart{f(I)}), h(\iend{f(I)})) \\
          & = \phi_{B,L}(h)(f(I)) \\
          & = \phi_{B,L}(h) \circ f(I) \\
          & = \aias(f,\inter[L])(\phi_{B,L}(h))(I)
      \end{align*}
    \item \textbf{right square}: Pick some $h : \slos(\points[A],L)$ and $I \in A$, then
      \begin{align*}
        \phi_{A,M}(\slos(\points[A],g)(h))(I)
          & = \phi_{A,M}(g \circ h)(I) \\
          & = (g(h(\istart{I})), g(h(\iend{I}))) \\
          & = (g(h(\istart{I})), g(h(\iend{I}))) \\
          & = \inter[g](h(\istart{I}), h(\iend{I})) \\
          & = \inter[g](\phi_{A,L}(h)(I)) \\
          & = \aias(A,\inter[g])(\phi{A,L}(h))(I)
      \end{align*}
  \end{itemize}

  % For an interval algebra $A$ and a strict linear order $L$, we have
  % \begin{equation*}
  %   \text{Hom}(\points[A], L) \cong \text{Hom}(A, \inter[L])
  % \end{equation*}
  % A map of linear orders from the start/end points of A to L, enforces a mapping of interval
  % algebras from A to the intervals of L, by checking where the start/end points of the interval end.

  % The above map is "canonical" so it should be natural in A and L.
\end{proof}

{\color{blue} TODO add example of this:

Notice that
\begin{equation*}
  \text{Hom}(\inter[L],A) \not\cong \text{Hom}(L, \points[A])
\end{equation*}
Since A does not necessarily have all the intervals given by $\points[A]$, so if we have a map on
the right, it isn't guaranteed that we have a corresponding map on the left.}

{\color{blue} TODO consider the unit and counit some more
\begin{itemize}
  \item counit should always be an isomorphism
  \item show cases when unit is not an isomorphism
\end{itemize}}


{\color{blue} TODO: consider what our functors do to elementary embeddings?}

\newpage
\section{The Model Theory of Interval Algebras}

\begin{thm}
  The class $\finslo$ of finite strict linear orders satisfies the hereditary property (HP),
  the joint embedding property (JEP), the amalgamation property (AP) and is essentially countable
  (EC). In other words, $\finslo$ is a Fraïssé class.
\end{thm}
\begin{proof}
  {\color{orange} TODO prove this theorem. Something like
  \begin{itemize}
    \item HP: Show the result for all relational, universal theories, since this is referenced in
      the later proof.
    \item JEP: Given two finite strict linear orders, embed them into their disjoint union with
      the left order befored the right order
    \item AP: Given two finite strict linear orders A and B, with C embedding into both,
      embed A and B into their product ordered lexicographically
    \item EC: Show the result for all finite languages, since this is referenced in the later proof.
  \end{itemize}}
\end{proof}

\begin{thm}
  The Fraïssé limit of $\finslo$ is $\Q$ with its usual order.
\end{thm}
\begin{proof}
  To show this, it suffices to show that $\Q$ is homogeneous and that its age is $\finslo$.

  First we show that $\Q$ is in fact homogeneous. For this, it is helpful to see how to expand the
  domain of an order isomorphism $f : L \to P$ with $L, P \subseteq \Q$ both finite. Suppose that we
  wish to extend the domain of $f$ to include some $a \in \Q \setminus L$. There are three cases to
  worry about here:
  \begin{itemize}
    \item If $a$ is an upper bound for $L$, we find some upper bound $b$ of $P$ which is
      not in $P$, such a $b$ must exist as $\Q$ is unbounded and $P$ is finite. Then we extend
      $f : L \to P$ to $g : L \cup \{a\} \to P \cup \{b\}$ by sending $g(a) = b$. This remains an
      order isomorphism since for any $x \in P$, we know that $x < a$ and $g(x) < b = g(a)$ since
      $a$ and $b$ are both upper bounds of $L$ and $P$ respectively.
    \item If $a$ is a lower bound for $L$, then we find a lower bound $b$ of $P$ not already in
      $P$ and extend $f$ to $g : L \cup \{a\} \to P \cup \{b\}$ by sending $a$ to $b$ again.
      Similarly to the upper bound case, since $a$ and $b$ are both lower bounds of the domain and
      codomain, this remains an order isomorphism.
    \item If $a$ is neither, then we notice that $L,P$ are finite linear orders and hence discrete.
      This means that we may find $a_1,a_2 \in L$ such that $a_1 < a < a_2$ and for no $x \in L$ do
      we have $a_1 < x < a_2$. Now we can find some $b \in \Q \setminus P$ such that
      $g(a_1) < b < g(a_2)$ since $\Q$ is dense and $P$ finite. We now extend $f$ to
      $g : L \cup \{a\} \to P \cup \{b\}$ by sending g(a) = b. This remains an order isomorphism
      since for any $x \in L$ we have either $x \leq a_1 < a$, so
      \begin{equation*}
        g(x) = f(x) \leq f(a_1) = g(a_1) < b = g(a)
      \end{equation*}
      or we have $a < a_2 \leq x$, in which case
      \begin{equation*}
        g(a) = b < g(a_2) = f(a_2) \leq f(x) = g(x)
      \end{equation*}
  \end{itemize}
  If we wanted to expand the codomain of $f$ in a similar way, we could just regard $f^{-1}$ as
  an order isomorphism between finite subsets of $\Q$, extend its domain to include whichever
  element we needed giving us a function $g$, then $g^{-1}$ would be the required extension of $f$.

  Now, fix two finite suborders $L,P \subseteq \Q$
  and suppose we have some order isomorphism $f : L \to P$. To extend $f$ to an automorphism of $\Q$
  we start by fixing an enumeration $(a_1, a_2, \dots)$ of the elements of $\Q$ and we define
  three sequences: $(L_1, L_2, \dots)$ and $(P_1, P_2, \dots)$ of increasing subsets of $\Q$ and
  $(g_1, g_2, \dots)$ of bijections $g_i : L_i \to P_i$ where each $g_i$ extends its predecessors.
  We define this sequences by induction:
  \begin{itemize}
    \item $k = 1$: let $L_1 = L,\ P_1 = P$ and $g_1 = f$.
    \item $k = 2l$ for $l \in \{1, 2, \dots\}$: at even indices we focus on increasing the domain of
      $g_i$ to all of $\Q$. If $a_l \in L_{k-1}$ then we let
      \begin{align*}
        L_k & = L_{k-1} \\
        P_k & = P_{k-1} \\
        g_k & = g_{k-1}
      \end{align*}
      Otherwise we extend $g_{k-1}$ to an order isomorphism
      $h : L_{k-1} \cup \{a_l\} \to P_{k-1} \cup \{b\}$ for some $b$ chosen appropriately and let
      \begin{align*}
        L_k & = L_{k-1} \cup \{a_l\} \\
        P_k & = P_{k-1} \cup \{b\} \\
        g_k & = h
      \end{align*}
    \item $k = 2l + 1$ for $l \in \{1, 2, \dots\}$: at odd indices we focus on increasing the
      codomain of $g_i$ to all of $\Q$. If $a_l \in P_{k-1}$ then we let
      \begin{align*}
        L_k & = L_{k-1} \\
        P_k & = P_{k-1} \\
        g_k & = g_{k-1}
      \end{align*}
      Otherwise we extend $g_{k-1}$ to an order isomorphism
      $h : L_{k-1} \cup \{b\} \to P_{k-1} \cup \{a_l\}$ for some $b$ chosen appropriately and let
      \begin{align*}
        L_k & = L_{k-1} \cup \{b\} \\
        P_k & = P_{k-1} \cup \{a_l\} \\
        g_k & = h
      \end{align*}
  \end{itemize}
  Finally, it is the case that $\Cup L_k = \Cup P_k = \Q$ since each $x \in \Q$ must appear as $a_l$
  in our enumeration for some $l \in \{1, 2, \dots\}$. Then $x \in L_{2l}$ and $x \in P_{2l+1}$ so
  definitely $x \in \Cup L_k$ and $x \in \Cup P_k$, so these unions must equal $\Q$. This means that
  we now have the function $g = \Cup g_k$ which extends $f$ by construction.

  To finish the proof we show that $\age{\Q} = \finslo$. Clearly $\age{\Q} \subseteq \finslo$ since
  a suborder of a linear order must still be linear. To see that $\finslo \subseteq \age{\Q}$ we fix
  some finite linear order $L$, then there is an order preserving isomorphism from $L$ to an initial
  segment of $\N$. The inclusion $\N \in \Q$ means this isomorphism realises $L$ as a finite
  suborder of $\Q$.
\end{proof}

\begin{thm}
  The class $\finaia$ of finite interval algebras is a Fraïssé class.
\end{thm}
\begin{proof}
  First, notice that $\points$ must send finite interval algebras to finite strict linear orders.
  In fact, given an interval algebra $A$, $|\points[A]| \leq 2|A|$ since
  $\points[A]$ is a quotient of $A + A$. Similarly, $\inter$ must also send finite strict linear
  orders to finite interval algebras: given a strict linear order $L$, $|\inter[L]| = |L|^2$.

  Next, with this in mind, we check the Fraïssé class conditions:
  \begin{itemize}
    \item \textbf{HP}: This follows from the fact that $\laia$ is a relational language and
      $\taia$ is a universal theory, same as the case for strict linear orders.
    \item \textbf{JEP}: Given two finite interval algebras $A$ and $B$, we use the JEP of strict
      linear orders to get the following diagram in $\slos$
      \[\begin{tikzcd}
        {\points[A]} \\
        && \Omega \\
        {\points[B]}
        \arrow["f", from=1-1, to=2-3]
        \arrow["g"', from=3-1, to=2-3]
      \end{tikzcd}\]
      Then, applying $\inter$ and using the adjunction unit $\eta$, we get
      % https://q.uiver.app/?q=WzAsNSxbMiwwLCJcXGludGVyW1xccG9pbnRzW0FdXSJdLFsyLDIsIlxcaW50ZXJbXFxwb2ludHNbQl1dIl0sWzQsMSwiXFxpbnRlcltcXE9tZWdhXSJdLFswLDAsIkEiXSxbMCwyLCJCIl0sWzAsMiwiXFxpbnRlcltmXSIsMCx7ImNvbG91ciI6WzAsNjAsNjBdfSxbMCw2MCw2MCwxXV0sWzEsMiwiXFxpbnRlcltnXSIsMix7ImNvbG91ciI6WzAsNjAsNjBdfSxbMCw2MCw2MCwxXV0sWzQsMSwiXFx1bml0e0J9IiwyXSxbMywwLCJcXHVuaXR7QX0iXV0=
      \[\begin{tikzcd}
        A && {\inter[\points[A]]} \\
        &&&& {\inter[\Omega]} \\
        B && {\inter[\points[B]]}
        \arrow["{\inter[f]}", color={rgb,255:red,214;green,92;blue,92}, from=1-3, to=2-5]
        \arrow["{\inter[g]}"', color={rgb,255:red,214;green,92;blue,92}, from=3-3, to=2-5]
        \arrow["{\unit{B}}"', from=3-1, to=3-3]
        \arrow["{\unit{A}}", from=1-1, to=1-3]
      \end{tikzcd}\]
      And the composites $\inter[f] \circ \unit{A}$ and $\inter[g] \circ \unit{B}$ along with
      the interval algebra $\Omega$ give us the joint embedding of $A$ and $B$.
    \item \textbf{AP}: Suppose we have the following diagram in $\aias$
      % https://q.uiver.app/?q=WzAsNCxbMiwwLCJBIl0sWzIsMiwiQiJdLFswLDEsIkMiXSxbNCwxXSxbMiwwLCJmIl0sWzIsMSwiZyIsMl1d
      \[\begin{tikzcd}
        && A \\
        C &&&& {} \\
        && B
        \arrow["f", from=2-1, to=1-3]
        \arrow["g"', from=2-1, to=3-3]
      \end{tikzcd}\]
      Applying $\points$ takes us to $\slos$, at which point we can use the AP of strict linear
      orders to get the commuting square
      % https://q.uiver.app/?q=WzAsNCxbMiwwLCJcXHBvaW50c1tBXSJdLFsyLDIsIlxccG9pbnRzW0JdIl0sWzAsMSwiXFxwb2ludHNbQ10iXSxbNCwxLCJcXE9tZWdhIl0sWzIsMCwiXFxwb2ludHNbZl0iLDAseyJjb2xvdXIiOlswLDYwLDYwXX0sWzAsNjAsNjAsMV1dLFsyLDEsIlxccG9pbnRzW2ddIiwyLHsiY29sb3VyIjpbMCw2MCw2MF19LFswLDYwLDYwLDFdXSxbMCwzLCJmJyJdLFsxLDMsImcnIiwyXV0=
      \[\begin{tikzcd}
        && {\points[A]} \\
        {\points[C]} &&&& \Omega \\
        && {\points[B]}
        \arrow["{\points[f]}", color={rgb,255:red,214;green,92;blue,92}, from=2-1, to=1-3]
        \arrow["{\points[g]}"', color={rgb,255:red,214;green,92;blue,92}, from=2-1, to=3-3]
        \arrow["{f'}", from=1-3, to=2-5]
        \arrow["{g'}"', from=3-3, to=2-5]
      \end{tikzcd}\]
      Now going back to $\aias$ gives the commuting diagram
      % https://q.uiver.app/?q=WzAsOSxbNCwwLCJcXGludGVyW1xccG9pbnRzW0FdXSJdLFs0LDIsIlxcaW50ZXJbXFxwb2ludHNbQl1dIl0sWzIsMSwiXFxpbnRlcltcXHBvaW50c1tDXV0iXSxbNiwxLCJcXGludGVyW1xcT21lZ2FdIl0sWzAsMSwiQyJdLFswLDBdLFsxLDBdLFsyLDAsIkEiXSxbMiwyLCJCIl0sWzIsMCwiXFxpbnRlcltcXHBvaW50c1tmXV0iLDEseyJjb2xvdXIiOlswLDYwLDYwXX0sWzAsNjAsNjAsMV1dLFsyLDEsIlxcaW50ZXJbXFxwb2ludHNbZ11dIiwxLHsiY29sb3VyIjpbMCw2MCw2MF19LFswLDYwLDYwLDFdXSxbMCwzLCJcXGludGVyW2YnXSIsMCx7ImNvbG91ciI6WzAsNjAsNjBdfSxbMCw2MCw2MCwxXV0sWzEsMywiXFxpbnRlcltnJ10iLDIseyJjb2xvdXIiOlswLDYwLDYwXX0sWzAsNjAsNjAsMV1dLFs4LDEsIlxcdW5pdHtCfSIsMl0sWzcsMCwiXFx1bml0e0F9Il0sWzQsMiwiXFx1bml0e0N9IiwxXSxbNCw3LCJmIl0sWzQsOCwiZyIsMl1d
      \[\begin{tikzcd}
        {} & {} & A && {\inter[\points[A]]} \\
        C && {\inter[\points[C]]} &&&& {\inter[\Omega]} \\
        && B && {\inter[\points[B]]}
        \arrow["{\inter[\points[f]]}"{description}, color={rgb,255:red,214;green,92;blue,92}, from=2-3, to=1-5]
        \arrow["{\inter[\points[g]]}"{description}, color={rgb,255:red,214;green,92;blue,92}, from=2-3, to=3-5]
        \arrow["{\inter[f']}", color={rgb,255:red,214;green,92;blue,92}, from=1-5, to=2-7]
        \arrow["{\inter[g']}"', color={rgb,255:red,214;green,92;blue,92}, from=3-5, to=2-7]
        \arrow["{\unit{B}}"', from=3-3, to=3-5]
        \arrow["{\unit{A}}", from=1-3, to=1-5]
        \arrow["{\unit{C}}"{description}, from=2-1, to=2-3]
        \arrow["f", from=2-1, to=1-3]
        \arrow["g"', from=2-1, to=3-3]
      \end{tikzcd}\]
      For the AP of the finite interval algebras, we are only interested in the outer square. The
      necessary maps are then $\inter[f'] \circ \unit{A}$ and $\inter[g'] \circ \unit{B}$, both
      mapping into $\inter[\Omega]$.
    \item \textbf{EC}: Follows from the fact that $\laia$ is finite, similarly to the strict linear
      order case.
\end{itemize}
\end{proof}

\begin{thm}
  The Fraïssé limit of $\finaia$ is $\inter[\Q]$
\end{thm}
\begin{proof}
  To show that the Fraïssé limit of the finite interval algebras is $\inter[\Q]$, it suffices to
  show the following:
  \begin{itemize}
    \item $\age{\inter[\Q]} = \finaia$: We show that both sides include into the other. First,
      suppose we have some finitely generated $\laia$-substructure $A \subset \inter[\Q]$. Since
      $\taia$ is universal, $A$ must also be an interval algebra. Furthermore, since $\laia$ is
      relational, $A$ must also be finite, so $A \in \finaia$. For the converse inclusion, consider
      some finite interval algebra $A$, it must embed into $\inter[\points[A]]$, which in turn
      embeds into $\inter[\Q]$ (since $\points[A]$ is finite, hence embeddable into $\Q$).
      Restricting these composition of these embeddings onto their image in $\inter[\Q]$ then gives
      the needed isomorphism.
    \item $\inter[\Q]$ is homogeneous: fix two $\laia$-substructures $A,B \subseteq \inter[\Q]$,
      along with some $\laia$-isomorphism $f : A \to B$. In essence we have the following
      diagram of interval algebras, where $i$ and $j$ are the inclusions into $\inter[\Q]$:
      % https://q.uiver.app/?q=WzAsNCxbMCwyLCJBIl0sWzIsMiwiQiJdLFswLDAsIlxcaW50ZXJbXFxRXSJdLFsyLDAsIlxcaW50ZXJbXFxRXSJdLFswLDEsImYiXSxbMCwyLCJpIl0sWzEsMywiaiIsMl1d
      \[\begin{tikzcd}
        {\inter[\Q]} && {\inter[\Q]} \\
        \\
        A && B
        \arrow["f", from=3-1, to=3-3]
        \arrow["i", from=3-1, to=1-1]
        \arrow["j"', from=3-3, to=1-3]
      \end{tikzcd}\]
      Applying $\inter$ to move to linear orders, we can postcompose $\points[i]$ and $\points[j]$ with
      the counit at $\Q$ to realise $\points[A]$ and $\points[B]$ as $\lslo$-substructures of $\Q$.
      Now, $\points[f]$ is still an isomorphism as these are preserved by functors, and using the
      fact that $\Q$ is homogeneous, we can extend $\points[f]$ to an isomorphism $g$, giving the
      commuting diagram
      % https://q.uiver.app/?q=WzAsNyxbMCw0LCJcXHBvaW50c1tBXSJdLFsyLDQsIlxccG9pbnRzW0JdIl0sWzAsMiwiXFxwb2ludHNbXFxpbnRlcltcXFFdXSJdLFsyLDIsIlxccG9pbnRzW1xcaW50ZXJbXFxRXV0iXSxbMCwwLCJcXFEiXSxbMiwwLCJcXFEiXSxbMCwxXSxbMCwxLCJcXHBvaW50c1tmXSIsMCx7ImNvbG91ciI6WzAsNjAsNjBdfSxbMCw2MCw2MCwxXV0sWzAsMiwiXFxwb2ludHNbaV0iLDAseyJjb2xvdXIiOlswLDYwLDYwXX0sWzAsNjAsNjAsMV1dLFsyLDQsIlxcY291bml0e1xcUX0iXSxbMyw1LCJcXGNvdW5pdHtcXFF9IiwyXSxbMSwzLCJcXHBvaW50c1tqXSIsMix7ImNvbG91ciI6WzAsNjAsNjBdfSxbMCw2MCw2MCwxXV0sWzQsNSwiZyJdXQ==
      \[\begin{tikzcd}
        \Q && \Q \\
        {} \\
        {\points[\inter[\Q]]} && {\points[\inter[\Q]]} \\
        \\
        {\points[A]} && {\points[B]}
        \arrow["{\points[f]}", color={rgb,255:red,214;green,92;blue,92}, from=5-1, to=5-3]
        \arrow["{\points[i]}", color={rgb,255:red,214;green,92;blue,92}, from=5-1, to=3-1]
        \arrow["{\counit{\Q}}", from=3-1, to=1-1]
        \arrow["{\counit{\Q}}"', from=3-3, to=1-3]
        \arrow["{\points[j]}"', color={rgb,255:red,214;green,92;blue,92}, from=5-3, to=3-3]
        \arrow["g", from=1-1, to=1-3]
      \end{tikzcd}\]
      Finally, we apply $\inter$ to bring us back to interval algebras, where we have the following
      diagram:
      % If modifying diagram, don't forget to add [scale cd=0.91]
      % https://q.uiver.app/?q=WzAsMTIsWzIsNCwiXFxpbnRlcltcXHBvaW50c1tBXV0iXSxbNCw0LCJcXGludGVyW1xccG9pbnRzW0JdXSJdLFsyLDIsIlxcaW50ZXJbXFxwb2ludHNbXFxpbnRlcltcXFFdXV0iXSxbNCwyLCJcXGludGVyW1xccG9pbnRzW1xcaW50ZXJbXFxRXV1dIl0sWzIsMCwiXFxpbnRlcltcXFFdIl0sWzQsMCwiXFxpbnRlcltcXFFdIl0sWzQsNiwiQiJdLFsyLDYsIkEiXSxbNiwyLCJcXGludGVyW1xcUV0iXSxbMCwyLCJcXGludGVyW1xcUV0iXSxbMCw0LCJBIl0sWzYsNCwiQiJdLFswLDEsIlxcaW50ZXJbXFxwb2ludHNbZl1dIiwwLHsiY29sb3VyIjpbMCw2MCw2MF19LFswLDYwLDYwLDFdXSxbMSwzLCJcXGludGVyW1xccG9pbnRzW2ldXSIsMCx7ImNvbG91ciI6WzAsNjAsNjBdfSxbMCw2MCw2MCwxXV0sWzAsMiwiXFxpbnRlcltcXHBvaW50c1tpXV0iLDIseyJjb2xvdXIiOlswLDYwLDYwXX0sWzAsNjAsNjAsMV1dLFsyLDQsIlxcaW50ZXJbXFxjb3VuaXR7XFxRfV0iLDIseyJjb2xvdXIiOlswLDYwLDYwXX0sWzAsNjAsNjAsMV1dLFszLDUsIlxcaW50ZXJbXFxjb3VuaXR7XFxRfV0iLDAseyJjb2xvdXIiOlswLDYwLDYwXX0sWzAsNjAsNjAsMV1dLFs0LDUsIlxcaW50ZXJbZ10iLDAseyJjb2xvdXIiOlswLDYwLDYwXX0sWzAsNjAsNjAsMV1dLFs3LDAsIlxcdW5pdHtBfSIsMl0sWzYsMSwiXFx1bml0e0J9Il0sWzgsMywiXFx1bml0e1xcaW50ZXJbXFxRXX0iXSxbOSwyLCJcXHVuaXR7XFxpbnRlcltcXFFdfSIsMl0sWzgsNSwiXFxpZHtcXGludGVyW1xcUV19IiwyLHsiY3VydmUiOjN9XSxbOSw0LCJcXGlke1xcaW50ZXJbXFxRXX0iLDAseyJjdXJ2ZSI6LTN9XSxbNyw2LCJmIiwyXSxbMTAsOSwiaSJdLFsxMCwwLCJcXHVuaXR7QX0iLDJdLFs3LDEwLCJcXGlke0F9IiwwLHsiY3VydmUiOi0zfV0sWzExLDgsImoiLDJdLFsxMSwxLCJcXHVuaXR7Qn0iXSxbNiwxMSwiXFxpZHtCfSIsMix7ImN1cnZlIjozfV1d
      \[\begin{tikzcd}[scale cd=0.91]
        && {\inter[\Q]} && {\inter[\Q]} \\
        \\
        {\inter[\Q]} && {\inter[\points[\inter[\Q]]]} && {\inter[\points[\inter[\Q]]]} && {\inter[\Q]} \\
        \\
        A && {\inter[\points[A]]} && {\inter[\points[B]]} && B \\
        \\
        && A && B
        \arrow["{\inter[\points[f]]}", color={rgb,255:red,214;green,92;blue,92}, from=5-3, to=5-5]
        \arrow["{\inter[\points[i]]}", color={rgb,255:red,214;green,92;blue,92}, from=5-5, to=3-5]
        \arrow["{\inter[\points[i]]}"', color={rgb,255:red,214;green,92;blue,92}, from=5-3, to=3-3]
        \arrow["{\inter[\counit{\Q}]}"', color={rgb,255:red,214;green,92;blue,92}, from=3-3, to=1-3]
        \arrow["{\inter[\counit{\Q}]}", color={rgb,255:red,214;green,92;blue,92}, from=3-5, to=1-5]
        \arrow["{\inter[g]}", color={rgb,255:red,214;green,92;blue,92}, from=1-3, to=1-5]
        \arrow["{\unit{A}}"', from=7-3, to=5-3]
        \arrow["{\unit{B}}", from=7-5, to=5-5]
        \arrow["{\unit{\inter[\Q]}}", from=3-7, to=3-5]
        \arrow["{\unit{\inter[\Q]}}"', from=3-1, to=3-3]
        \arrow["{\id{\inter[\Q]}}"', curve={height=18pt}, from=3-7, to=1-5]
        \arrow["{\id{\inter[\Q]}}", curve={height=-18pt}, from=3-1, to=1-3]
        \arrow["f"', from=7-3, to=7-5]
        \arrow["i", from=5-1, to=3-1]
        \arrow["{\unit{A}}"', from=5-1, to=5-3]
        \arrow["{\id{A}}", curve={height=-18pt}, from=7-3, to=5-1]
        \arrow["j"', from=5-7, to=3-7]
        \arrow["{\unit{B}}", from=5-7, to=5-5]
        \arrow["{\id{B}}"', curve={height=18pt}, from=7-5, to=5-7]
      \end{tikzcd}\]
      Although not obvious at first, the above diagram commutes, to check this we look at all the
      "irreducible components" individually:
      \begin{itemize}
        \item The middle rectangle in red commutes since it already commuted for linear orders.
        \item The top triangles commute by triangle identities of our adjunction.
        \item The squares to the left, right and bottom of the red rectangle commute by naturality
          of the unit.
        \item The bottom triangles commute due to the use of the identity.
      \end{itemize}
      Chasing around the outside of the diagram, we see that
      \begin{equation*}
        \inter[g] \circ \id{\inter[\Q]} \circ i \circ \id{A} =
          \id{\inter[\Q]} \circ j \circ \id{B} \circ f
      \end{equation*}
      Simplifying by removing identities shows that $\inter[g] \circ i = j \circ f$. Now, since $g$
      was an isomorphism, so is $\inter[g]$, so we have successfully extended $f$ to an
      automorphism of $\inter[\Q]$.
  \end{itemize}
\end{proof}

{\color{blue} TODO Fraisse Limits can be seen as a type of colimit, check if there is anything
interesting/noteworthy to say here -- left adjoints only preserve limits generally, but here we
have a colimit being preserved by a left adjoint (Int(-))}
% Fraisse Limits can be seen as a type of colimit, look into this?
% It's interesting that this colimit is preserved by a right adjoint, namely Int(-)
% WARNING: Not sure if the following makes any sense, I am quite tired -- I suspect my reasoning is
% somewhat flawed though?
% I guess it happens since Pts(Int(-)) is naturally isomorphic to Id(-),
% So suppose the colimit of some diagram D(-) exists for linear orders,
% then if the colimit of Pts(D(-)) exists, it must be isomorphic to the Pts(colimit of D(-))
% This is because left adjoints preserve colimits, so Int(colimit of Pts(D(-))) is the colimit of
% Int(Pts(D(-))), and since Ints(Pts(D(-))) is naturally isomorphic to D(-), the colimit of
% Int(Pts(D(-))) must also be isomorphic to the colimit of D(-)

% ------ Stability theory

\begin{thm}
  An interval algebra $A$ is stable if and only if it is finite.
\end{thm}
\begin{proof}
  % TODO look at exercise 5.5.6 of marker for the order property.
  Suppose that $A$ is an interval algebra and consider the formula
  \begin{equation*}
    \lexord(I, J) = (I \before J) \lor (I \meets J) \lor (I \overlaps J) \lor (I \starts J)
                                  \lor (I \finishedby J) \lor (I \contains J)
  \end{equation*}
  The above formula takes the elements of $A$ and lexicographically orders them, by first comparing
  the start times of each interval, followed by the end times in case the start times align. As a
  result, the formula $\lexord$ induces the structure of a strict linear order on the elements
  of $A$:
  \begin{itemize}
    \item \textbf{irreflexivity}: Fix some interval $I \in A$, then $I = I$. Since the relational
      symbols of interval algebras along with equality are all mutually exclusive,
      neither $I \aiaarrow{i} I$ nor $I \raiaarrow{i} I$ can hold for any $i \in \aiaindex$. This
      means $A \models \lnot \lexord(I, I)$ so $\lexord$ is irreflexive.
    \item \textbf{transitivity}: Fix three intervals $I,J,K \in A$ and suppose that
      $A \models \lexord(I,J)$ and $A \models \lexord(J,K)$. Since $\lexord$ consists of the disjunction of 6
      relational symbols, there are 36 cases which would lead to this situation. Considering each of
      these cases individually and looking up our transitivity axioms for interval algebras, we can
      find all possible relations between $I$ and $K$, which is detailed in table
      \ref{tab:lexord_trans}. In all of these cases, we must still have $A \models \lexord(I,K)$, so
      $\lexord$ is transitive.
      \transreltable{lexord_trans}{Cases for transitivy of $\lexord$.}{
        \llrow & \slrow \\
        \lmrow & \smrow \\
        \lorow & \sorow \\
        \lsrow & \ssrow \\
        \lFrow & \sFrow \\
        \lDrow & \sDrow \\
        \hline
        \mlrow & \Flrow \\
        \mmrow & \Fmrow \\
        \morow & \Forow \\
        \msrow & \Fsrow \\
        \mFrow & \FFrow \\
        \mDrow & \FDrow \\
        \hline
        \olrow & \Dlrow \\
        \omrow & \Dmrow \\
        \oorow & \Dorow \\
        \osrow & \Dsrow \\
        \oFrow & \DFrow \\
        \oDrow & \DDrow \\
      }
    \item \textbf{trichotomy}: Fix two intervals $I, J \in A$. First notice that
      \begin{equation*}
        A \models \lexord(J,I) \iff
        (I \after J) \lor (I \metby J) \lor (I \overlappedby J) \lor (I \startedby J)
                      \lor (I \finishes J) \lor (I \contained J)
      \end{equation*}
      Hence $A \models \lexord(I,J) \lor (I = J) \lor \lexord(J,I)$ is equivalent to saying that
      the interval algebra relations are exhaustive, which is one of our axioms. Hence the strict
      ordering given by $\lexord$ is linear.
  \end{itemize}

  Now for any finite interval algebra $A$, all strict linear orders interpretable in $A$ will be
  bounded in size by $|A^k|$ for some $k \in \N$. In particular, all such linear orders must be
  finite, so $A$ is stable.

  For the converse, suppose $A$ is stable. Using $\lexord$ we can linearly order all the elements
  of $A$, but as $A$ is stable this linear order must be finite, hence $A$ must be finite.
\end{proof}

\begin{thm}
  If a model M is interpretable in an NIP model, then M must also be NIP
\end{thm}
\begin{proof}
  {\color{orange} Prove the contrapositive (if M is IP, then anything M is interpretable in is IP)
  If M is interpretable in $N^k$, and there is a formula $\phi$ which has the IP in M,
  then viewing this formula as a formula over $N$ implies that $N$ is also IP.}
\end{proof}

\begin{cor}
  Given an NIP linear order $L$, the interval algebra $\inter[L]$ is also NIP.
\end{cor}
\begin{proof}
  {\color{orange} TODO Prove this.

  $\inter[L]$ is interpretable in $L$ by its construction.}
\end{proof}

\begin{thm}
  Given an linear order $L$ with no upper bound (or no lower bound), $L$ is an NIP model if and only
  if $\inter[L]$ is an NIP model.
\end{thm}
\begin{proof}
  {\color{orange} TODO Prove this.

  The left to right implication holds by the previous corollary.
  For the right to left implication, we notice that we can interpret the linear order $L$ in
  $\inter[L]$ as it has no upper bound (use a construction similar to Points where we only consider
  the end points of an interval -- the lower bound analogue follows from the start point
  construction).}
\end{proof}

\end{document}
