\documentclass[11pt % Font size
              ]{article}
\usepackage[ a4paper % Paper size and format
           , onecolumn % Number of columns: onecolumn or twocolumn
           , total={6in, 8in} % Text area: {width, height}
           ]{geometry}

\title{A Model Theoretic Study of Allen's Interval Algebra}
\author{Bruno da Rocha Paiva - bmd18}
\date{} % Date of writing: empty, today or 25.12.00

\usepackage{amsmath}
\usepackage{amsthm}
\usepackage{amssymb}
\usepackage{parskip}

\usepackage{subfiles}

% General Symbols
\newcommand{\Q}{\mathbb{Q}}
\newcommand{\R}{\mathbb{R}}

% Model Theoretic Symbols
\newcommand{\lang}{\mathcal{L}}
\newcommand{\theory}{\mathbb{T}}

% Allen Interval Algebra Symbols
\newcommand{\laia}{\lang_\text{AIA}}
\newcommand{\taia}{\theory_\text{AIA}}
%\renewcommand{\int}{\text{Inter}}
\newcommand{\inter}[1]{\text{Inter}\left(#1\right)}
%\newcommand{\points}{\text{Points}}
\newcommand{\points}[1]{\text{Points}\left(#1\right)}
\newcommand{\intstart}[1]{#1_{\text{start}}}
\newcommand{\intend}[1]{#1_{\text{end}}}

\newcommand{\lslo}{\lang_\text{SLO}}
\newcommand{\tslo}{\theory_\text{SLO}}

% Allen Interval Algebra Relations
\newcommand{\aiaarrow}[1]{\overset{#1}{\longrightarrow}}
\newcommand{\before}{\overset{<}{\longrightarrow}}
\newcommand{\meets}{\overset{m}{\longrightarrow}}
\newcommand{\overlaps}{\overset{o}{\longrightarrow}}
\newcommand{\starts}{\overset{s}{\longrightarrow}}
\newcommand{\finishes}{\overset{f}{\longrightarrow}}
\newcommand{\contained}{\overset{d}{\longrightarrow}}
\newcommand{\after}{\overset{<}{\longleftarrow}}
\newcommand{\metby}{\overset{m}{\longleftarrow}}
\newcommand{\overlappedby}{\overset{o}{\longleftarrow}}
\newcommand{\startedby}{\overset{s}{\longleftarrow}}
\newcommand{\finishedby}{\overset{f}{\longleftarrow}}
\newcommand{\contains}{\overset{d}{\longleftarrow}}

% Environments
\theoremstyle{plain}
\newtheorem{thm}{Theorem}%[subsection]
\newtheorem{lemma}[thm]{Lemma}
\newtheorem{prop}[thm]{Proposition}
\newtheorem{cor}[thm]{Corollary}

\theoremstyle{definition}
\newtheorem{defn}[thm]{Definition}

\theoremstyle{remark}
\newtheorem*{rem}{Remark}

\begin{document}

\maketitle

\begin{defn}
  We define the language of strict linear orders $\lslo$ as the single binary relation $\{ < \}$.

  We define the theory of strict linear orders as
  \begin{align*}
    \tslo = \{ & \forall a, \lnot\ a < a, \\
              & \forall a, \forall b, \forall c, a < b \land b < c \rightarrow a < c \\
              & \forall a, \forall b, a < b \lor a = b \lor b < a \}
  \end{align*}
\end{defn}

\begin{defn}
  We define the language of Allen interval algebras $\laia$ as 
  \begin{equation*}
    \laia = \{ \before, \meets, \overlaps, \starts, \finishes, \contained,
               \after, \metby, \overlappedby, \startedby, \finishedby, \contains \}
  \end{equation*}

  (Unsure on how to best write out the definition of $\taia$, especially with the "transitivity"
  axioms).
\end{defn}

% ------ Justifying this axiomatisation of interval algebras

\begin{defn}
  Given a linear order $L$ we define its set of non-zero intervals $\inter{P}$ as the set
  \begin{equation*}
    \inter{L} = \left\{(x_1,x_2)\ |\ x_1 < x_2 \right\} \subseteq L^2
  \end{equation*}
  We can turn this into a $\laia$-structure under the interpretations:
  \begin{itemize}
    \item $(x_1,x_2) \before (y_1,y_2)$ if $x_1<x_2<y_1<y_2$;
    \item $(x_1,x_2) \meets (y_1,y_2)$ if $x_1<x_2=y_1<y_2$;
    \item $(x_1,x_2) \overlaps (y_1,y_2)$ if $x_1<y_1<x_2<y_2$;
    \item $(x_1,x_2) \starts (y_1,y_2)$ if $x_1=y_1<x_2<y_2$;
    \item $(x_1,x_2) \finishes (y_1,y_2)$ if $y_1<x_1<x_2=y_2$;
    \item $(x_1,x_2) \contained (y_1,y_2)$ $y_1<x_1<x_2<y_2$;
    \item $(x_1,x_2) \overset{i}{\longleftarrow} (y_1,y_2)$ if and only if
      $(y_1,y_2) \overset{i}{\longrightarrow} (x_1,x_2)$ for $i \in \{<,m,o,s,f,d\}$.
  \end{itemize}
\end{defn}

\begin{thm}
  Given a strict linear order $L$, $\inter{L}$ is a model of $\taia$ under the above
  interpretations.
\end{thm}
\begin{proof}
  Tedious proof by cases goes here.
\end{proof}

\begin{cor}
  Allen's interval algebras are satisfiable
\end{cor}

The theory is satisfiable in a reasonable way, inducing the expected models from linear orders.
To further enforce this tight connection, we see interval algebras also induce a quite reasonable
linear order

\begin{defn}
  Given an Allen interval algebra $A$, we define $\points{A} = \frac{A + A}{\sim}$ where
  $\sim$ is an equivalence relation on the set $A + A$ given by:
  \begin{align*}
    (0,I) \sim (0,J) & \iff I \starts J \lor I \startedby J \lor I = J \\
    (1,I) \sim (1,J) & \iff I \finishes J \lor I \finishedby J \lor I = J \\
    (1,I) \sim (0,J) & \iff I \meets J \\
    (0,I) \sim (1,J) & \iff I \metby J
  \end{align*}
\end{defn}

\begin{thm}
  Given an Allen interval algebra $A$, the interpretation of the symbol $<$ in $\points{A}$ given by
  \begin{align*}
    (0,I) < (0,J) & \iff I \before J \lor I \meets J \lor I \overlaps J 
      \lor I \finishedby J \lor I \contains J \\
    (1,I) < (1,J) & \iff I \before J \lor I \meets J \lor I \overlaps J
      \lor I \starts J \lor I \contained J \\
    (1,I) < (0,J) & \iff I \before J \\
    (0,I) < (1,J) & \iff I \after J
  \end{align*}
  is well-defined and turns $A$ into a model of strict linear orders.
\end{thm}
\begin{proof}
  Another proof by too many cases.
\end{proof}

Now that this axiomatisation is justified, show that we can't remove any axioms without making
it into something we do not want.

% ------ A slight detour into category theory

\begin{defn}
  Given a theory $\theory$ over language $\lang$, we denote by $\text{Mod}(\lang, \theory)$ the category
  with objects the models of $\theory$ and arrows the $\lang$-embeddings.
\end{defn}

\begin{thm}
  We can turn $\text{Inter}$ into a functor from $\text{Mod}(\lslo,\tslo)$ into
  $\text{Mod}(\laia,\taia)$ by sending arrows $f : M \to N$ in $\text{Mod}(\lslo,\tslo)$ to the arrow
  $\inter{f} : \inter{M} \to \inter{N}$ in $\text{Mod}(\laia,\taia)$ defined by
  \begin{equation*}
    \inter{f}(x_1, x_2) = (f(x_1), f(x_2))
  \end{equation*}
\end{thm}
\begin{proof}
  Check the map preserves interpretations.

  Standard checking of functor axioms.
\end{proof}

\begin{thm}
  We can turn $\text{Points}$ into a functor from $\text{Mod}(\laia,\taia)$ into
  $\text{Mod}(\lslo,\tslo)$ by sending arrows $f : A \to B$ in $\text{Mod}(\laia,\taia)$ to the arrow
  $\points{f} : \points{A} \to \points{B}$ in $\text{Mod}(\lslo,\tslo)$ defined by
  \begin{equation*}
    \points{f}(0, I) = (0, f(I)) \qquad\text{and}\qquad \points{f}(1,I) = (1, f(I))
  \end{equation*}
\end{thm}
\begin{proof}
  Check map is monotone.

  Standard checking of functor axioms.
\end{proof}

\begin{thm}
  $\text{Points}$ is left adjoint to $\text{Inter}$.
\end{thm}
\begin{proof}
  For an interval algebra $A$ and a strict linear order $L$, we have
  \begin{equation*}
    \text{Hom}(\points{A}, L) \cong \text{Hom}(A, \inter{L})
  \end{equation*}
  A map of linear orders from the start/end points of A to L, enforces a mapping of interval
  algebras from A to the intervals of L, by checking where the start/end points of the interval end.
 
  The above map is "canonical" so it should be natural in A and L.

  Notice also that
  \begin{equation*}
    \text{Hom}(\inter{L},A) \not\cong \text{Hom}(L, \points{A})
  \end{equation*}
  Since A does not necessarily have all the intervals given by $\points{A}$, so if we have a map on
  the right, it isn't guaranteed that we have a corresponding map on the left.
\end{proof}

% ------ Homogeneous tings eh

\begin{thm}
  The class $\mathbf{FCh}$ of finite strict linear orders satisfies the hereditary property (HP),
  the joint embedding property (JEP), the amalgamation property (AP) and is essentially countable
  (EC). In other words, $\textbf{FCh}$ is a Fraïssé class.
\end{thm}
\begin{proof}
  Something like
  \begin{itemize}
    \item HP: a suborder of a finite strict linear order must also be finite and strictly linear
    \item JEP: Given two finite strict linear orders, embed them into their disjoint union with
      the left order befored the right order
    \item AP: Given two finite strict linear orders A and B, with C embedding into both,
      embed A and B into their product ordered lexicographically
    \item EC: For each natural n, there exists exactly one linear order of size n up to iso, so
      up to iso the number of finite linear orders must be countable
  \end{itemize}
\end{proof}

\begin{thm}
  The Fraïssé limit of $\textbf{FCh}$ is $\mathbb{Q}$
\end{thm}
\begin{proof}
  Show that $\mathbb{Q}$ is homogeneous and that its age is $\textbf{FCh}$.
\end{proof}

\begin{thm}
  The class of finite interval algebras is a Fraïssé class.
\end{thm}
\begin{proof}
  Notice $\text{Inter}$ and $\text{Points}$ must send finite models to finite models.

  For HP, JEP and AP, we can reduce to the case of linear orders by using the unit of the
  adjunction between interval algebras and linear orders.

  For EC use a similar argument to linear orders.
\end{proof}

\begin{thm}
  The Fraïssé limit of the finite interval algebras is $\inter{\mathbb{Q}}$
\end{thm}
\begin{proof}
  Show that the age of $\inter{\mathbb{Q}}$ is the finite interval algebras.

  Show that $\inter{\mathbb{Q}}$ is homogeneous by reducing to $\mathbb{Q}$ using our adjunction.

  (Given an isomorphism between finite interval algebras f, get an isomorphism between finite linear
  orders Points(f), this extends to an automorphism of Q as a linear order, which we bring over
  to Inter(Q)).
\end{proof}


% ------ Stability theory

\begin{thm}
  An interval algebra $A$ is stable if and only if it is finite.
\end{thm}
\begin{proof}
  Finite models are always stable since they cannot have infinitely many elements.

  For an infinite model $A$, we can linearly order the intervals in $A$, showing $A$ is unstable.
\end{proof}

\begin{thm}
  If a model M is interpretable in an NIP model, then M must also be NIP
\end{thm}
\begin{proof}
  Prove the contrapositive (if M is IP, then anything M is interpretable in is IP)
  If M is interpretable in $N^k$, and there is a formula $\phi$ which has the IP in M,
  then viewing this formula as a formula over $N$ implies that $N$ is also IP.
\end{proof}

\begin{cor}
  Given an NIP linear order $L$, the interval algebra $\inter{L}$ is also NIP.
\end{cor}
\begin{proof}
  $\inter{L}$ is interpretable in $L$ by its construction.
\end{proof}

\begin{thm}
  Given an linear order $L$ with no upper bound (or no lower bound), $L$ is an NIP model if and only
  if $\inter{L}$ is an NIP model.
\end{thm}
\begin{proof}
  The left to right implication holds by the previous corollary.
  For the right to left implication, we notice that we can interpret the linear order $L$ in
  $\inter{L}$ as it has no upper bound (use a construction similar to Points where we only consider
  the end points of an interval -- the lower bound analogue follows from the start point
  construction).
\end{proof}

\end{document}
