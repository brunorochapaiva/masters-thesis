% 0 0 0 case
\begin{table}[ht]
  \centering
  \begin{tabular}{| c | c | c || c | c | c || c | c | c |}
    \hline
    $I \to J$ & $J \to K$ & $I \to K$ &
      $I \to J$ & $J \to K$ & $I \to K$ &
      $I \to J$ & $J \to K$ & $I \to K$ \\
    \hline\hline
    \llrow & \olrow & \Dlrow \\
    \lmrow & \omrow & \Dmrow \\
    \lorow & \oorow & \Dorow \\
    \lFrow & \oFrow & \DFrow \\
    \lDrow & \oDrow & \DDrow \\
    \hline
    \mlrow & \Flrow &&&\\
    \mmrow & \Fmrow &&&\\
    \morow & \Forow &&&\\
    \mFrow & \FFrow &&&\\
    \mDrow & \FDrow &&&\\
    \hline
  \end{tabular}
  \caption{
    The transitivity table for the $[(0,I)] < [(0,J)]$ and $[(0,J)] < [(0,K)]$ case.
    For this to imply $[(0,I)] < [(0,K)]$ we need the $I \to K$ columns to all contain a
    subset of the string $<moFD$.
  }
  \label{tab:plt_trans_000}
\end{table}

% 0 0 1 case
\begin{table}[ht]
  \centering
  \begin{tabular}{| c | c | c || c | c | c || c | c | c |}
    \hline
    $I \to J$ & $J \to K$ & $I \to K$ &
      $I \to J$ & $J \to K$ & $I \to K$ &
      $I \to J$ & $J \to K$ & $I \to K$ \\
    \hline\hline
    \llrow & \olrow & \Dlrow \\
    \lmrow & \omrow & \Dmrow \\
    \lorow & \oorow & \Dorow \\
    \lsrow & \osrow & \Dsrow \\
    \lfrow & \ofrow & \Dfrow \\
    \ldrow & \odrow & \Ddrow \\
    \lerow & \oerow & \Derow \\
    \lOrow & \oOrow & \DOrow \\
    \lSrow & \oSrow & \DSrow \\
    \lFrow & \oFrow & \DFrow \\
    \lDrow & \oDrow & \DDrow \\
    \hline
    \mlrow & \Flrow &&&\\
    \mmrow & \Fmrow &&&\\
    \morow & \Forow &&&\\
    \msrow & \Fsrow &&&\\
    \mfrow & \Ffrow &&&\\
    \mdrow & \Fdrow &&&\\
    \merow & \Ferow &&&\\
    \mOrow & \FOrow &&&\\
    \mSrow & \FSrow &&&\\
    \mFrow & \FFrow &&&\\
    \mDrow & \FDrow &&&\\
    \hline
  \end{tabular}
  \caption{
    The transitivity table for the $[(0,I)] < [(0,J)]$ and $[(0,J)] < [(1,K)]$ case.
    For this to imply $[(0,I)] < [(1,K)]$ we need the $I \to K$ columns to all contain a
    subset of the string $<mosfd=OSFD$. Recall that concur is shorthand for $osfd=OSFD$
  }
  \label{tab:plt_trans_001}
\end{table}


% 0 1 0 case
\begin{table}[ht]
  \centering
  \begin{tabular}{| c | c | c |}
    \hline
    $I \to J$ & $J \to K$ & $I \to K$ \\
    \hline\hline
    \llrow \\
    \mlrow \\
    \olrow \\
    \slrow \\
    \flrow \\
    \dlrow \\
    \elrow \\
    \Olrow \\
    \Slrow \\
    \Flrow \\
    \Dlrow \\
    \hline
  \end{tabular}
  \caption{
    The transitivity table for the $[(0,I)] < [(1,J)]$ and $[(1,J)] < [(0,K)]$ case.
    For this to imply $[(0,I)] < [(0,K)]$ we need the $I \to K$ columns to all contain a
    subset of the string $<mofD$.
  }
  \label{tab:plt_trans_010}
\end{table}

% 0 1 1 case
\begin{table}[ht]
  \centering
  \begin{tabular}{| c | c | c || c | c | c || c | c | c |}
    \hline
    $I \to J$ & $J \to K$ & $I \to K$ &
      $I \to J$ & $J \to K$ & $I \to K$ &
      $I \to J$ & $J \to K$ & $I \to K$ \\
    \hline\hline
    \llrow & \flrow & \Slrow \\
    \lmrow & \fmrow & \Smrow \\
    \lorow & \forow & \Sorow \\
    \lsrow & \fsrow & \Ssrow \\
    \ldrow & \fdrow & \Sdrow \\
    \hline
    \mlrow & \dlrow & \Flrow \\
    \mmrow & \dmrow & \Fmrow \\
    \morow & \dorow & \Forow \\
    \msrow & \dsrow & \Fsrow \\
    \mdrow & \ddrow & \Fdrow \\
    \hline
    \olrow & \elrow & \Dlrow \\
    \omrow & \emrow & \Dmrow \\
    \oorow & \eorow & \Dorow \\
    \osrow & \esrow & \Dsrow \\
    \odrow & \edrow & \Ddrow \\
    \hline
    \slrow & \Olrow &&&\\
    \smrow & \Omrow &&&\\
    \sorow & \Oorow &&&\\
    \ssrow & \Osrow &&&\\
    \sdrow & \Odrow &&&\\
    \hline
  \end{tabular}
  \caption{
    The transitivity table for the $[(0,I)] < [(1,J)]$ and $[(1,J)] < [(1,K)]$ case.
    For this to imply $[(0,I)] < [(1,K)]$ we need the $I \to K$ columns to all contain a
    subset of the string $<mosfd=OSFD$. Recall that concur is shorthand for $osfd=OSFD$.
  }
  \label{tab:plt_trans_011}
\end{table}

% 1 0 0 case
\begin{table}[ht]
  \centering
  \begin{tabular}{| c | c | c |}
    \hline
    $I \to J$ & $J \to K$ & $I \to K$ \\
    \hline\hline
    \llrow \\
    \lmrow \\
    \lorow \\
    \lFrow \\
    \lDrow \\
    \hline
  \end{tabular}
  \caption{
    The transitivity table for the $[(1,I)] < [(0,J)]$ and $[(0,J)] < [(0,K)]$ case.
    For this to imply $[(1,I)] < [(0,K)]$ we need the $I \to K$ columns to all contain $<$.
  }
  \label{tab:plt_trans_100}
\end{table}

% 1 0 1 case
\begin{table}[ht]
  \centering
  \begin{tabular}{| c | c | c |}
    \hline
    $I \to J$ & $J \to K$ & $I \to K$ \\
    \hline\hline
    \llrow \\
    \lmrow \\
    \lorow \\
    \lsrow \\
    \lfrow \\
    \ldrow \\
    \lerow \\
    \lOrow \\
    \lSrow \\
    \lFrow \\
    \lDrow \\
    \hline
  \end{tabular}
  \caption{
    The transitivity table for the $[(1,I)] < [(0,J)]$ and $[(0,J)] < [(1,K)]$ case.
    For this to imply $[(1,I)] < [(1,K)]$ we need the $I \to K$ columns to all contain a
    subset of the string $<mosd$.
  }
  \label{tab:plt_trans_101}
\end{table}

% 1 1 0 case
\begin{table}[ht]
  \centering
  \begin{tabular}{| c | c | c |}
    \hline
    $I \to J$ & $J \to K$ & $I \to K$ \\
    \hline\hline
    \llrow \\
    \mlrow \\
    \olrow \\
    \slrow \\
    \dlrow \\
    \hline
  \end{tabular}
  \caption{
    The transitivity table for the $[(1,I)] < [(1,J)]$ and $[(1,J)] < [(0,K)]$ case.
    For this to imply $[(1,I)] < [(0,K)]$ we need the $I \to K$ columns to all contain $<$.
  }
  \label{tab:plt_trans_110}
\end{table}

% 1 1 1 case
\begin{table}[ht]
  \centering
  \begin{tabular}{| c | c | c || c | c | c || c | c | c |}
    \hline
    $I \to J$ & $J \to K$ & $I \to K$ &
      $I \to J$ & $J \to K$ & $I \to K$ &
      $I \to J$ & $J \to K$ & $I \to K$ \\
    \hline\hline
    \llrow & \olrow & \dlrow \\
    \lmrow & \omrow & \dmrow \\
    \lorow & \oorow & \dorow \\
    \lsrow & \osrow & \dsrow \\
    \ldrow & \odrow & \ddrow \\
    \hline
    \mlrow & \slrow &&&\\
    \mmrow & \smrow &&&\\
    \morow & \sorow &&&\\
    \msrow & \ssrow &&&\\
    \mdrow & \sdrow &&&\\
    \hline
  \end{tabular}
  \caption{
    The transitivity table for the $[(1,I)] < [(1,J)]$ and $[(1,J)] < [(1,K)]$ case.
    For this to imply $[(1,I)] < [(1,K)]$ we need the $I \to K$ columns to all contain a
    subset of the string $<mosd$.
  }
  \label{tab:plt_trans_111}
\end{table}