\section{Background}%
\label{sec:background}

\subsection{Allen's Interval Algebra}%
\label{sub:allen_interval_algebras}

The concept of Allen's interval algebras was first introduced in \cite{allen83}. The main
idea was to introduce a new system for arguing about time intervals in a qualitative way, similar
to how humans think about time. When one relates to someone else how some events might have
happened, it is rarely done with specific figures in mind. For example, when retelling a story
people will convey some ordering of events' start and end times, but skipping over the details
of the figures. Now, this might happen because the specific time frames don't add to the
story, or even because they are not known. Regardless of the reason, it is often helpful to argue
about time without reference to specific times. Computers however, tend to argue about time in a
quantitative manner: saving timestamps along with logs, such that if something goes wrong, the
order of events can be gotten; checking the system clock to tell if access tokens have expired;
so on. The difficulties of expressing time qualitatively on a computer become even more obvious
when working towards artificial intelligence, like in natural language processing, where the
computer must understand the timing of events from people's informal speech or writing.

Interval algebras were Allen's approach to have computers reasoning about time as humans did and
still do. This was done by treating time intervals as a primitive notion, with binary relations
recording their ordering and level of overlap. Allen's interval algebra consists of 13 basic
relations which allow one to describe fully how any two intervals $I$ and $J$ might relate. The
relations and their meaning can be found on table \ref{tab:basic_relations}: relations come in
pairs (apart from $=$) since for each relation, one can get its dual (or opposite) relation.
If it is known that \(I < J\), then it can be immediately infered that \(J > I\). Due to this
duality, the meaning of the opposite relations has been omitted to save space, although it can be
got by swapping the roles of $I$ and $J$. We have also used $I-$ and $I+$ as shorthands for the
start and end points of the interval $I$.

\begin{table}[htpb]
  \centering
  \begin{tabular}{|l|c|c|c|}
    \hline
    Relation & Symbol & Dual Symbol & Meaning \\
    \hline
    \(I\) starts \(J\)
             & \(I < J\)
             & \(I > J\)
             & \(I- < I+ < J- < J+\)\\
    \(I\) meets \(J\)
             & \(I\text{ m }J\)
             & \(I\text{ M }J\)
             & \(I- < I+ = J - < J+\)\\
    \(I\) overlaps \(J\)
             & \(I\text{ o }J\)
             & \(I\text{ O }J\)
             & \(I- < J- < I+ < J+\)\\
    \(I\) starts \(J\)
             & \(I\text{ s }J\)
             & \(I\text{ S }J\)
             & \(J- < I- < I+ < J+\)\\
    \(I\) finishes \(J\)
             & \(I\text{ f }J\)
             & \(I\text{ F }J\)
             & \(I- = J- < I+ < J+\)\\
    \(I\) during \(J\)
             & \(I\text{ d }J\)
             & \(I\text{ D }J\)
             & \(J- < I- < I+ = J+\)\\
    \(I\) equals \(J\)
             & \(I = J\)
             & \(I = J\)
             & \(I- = J- < I+ = J+\)\\
    \hline
  \end{tabular}
  \caption{13 Basic Relations of Allen's Interval Algebra}
  \label{tab:basic_relations}
\end{table}

The choice of these 13 relations has some advantages which simplify reasoning: this set of
relations is both exhaustive and mutually-exclusive. The first means that for any two intervals
$I$ and $J$, there exists at least one relation R  such that $I\text{ R }J$. The latter
means there exists at most one such relation. These facts will be important when coming up
with a first-order axiomatisation of Allen's interval algebras, so it will be helpful to keep them
in mind going forward.

Reading the "Meaning" column of table \ref{tab:basic_relations}, it seems to be the case that
no time points are not considered as valid intervals, since $I-$ is always strictly less than
$I+$. This is no accident, as time points can cause ambiguities and thus complicate the
semantics of interval algebras. This disallowance can be problematic too though, since the
concept of points in time is often used without thought in natural language. For example, the
phrase "I caught the ball" suggests that the act of catching the ball was instantaneous, and in a
sense it was, at least compared to the timescales used in everyday language. On the other hand,
the sentence "I met with them yesterday" clearly gives an interval of 24 hours when this meeting
might have taken place. In reality, what is or is not a time point depends a lot on the
context of the sentence: in a history lesson, it is entirely reasonable to think of days as
points, but describe to someone the big bang and nanoseconds suddenly become long intervals of
time. This reliance on context should evidence that interval algebras do not need the concept
of time points and that these can be modelled by suitably intervals.

\subsubsection{Basic Algorithm}%
\label{ssub:basic_algorithm}

With some of the main ideas explained, now is a good time to see the basic algorithm described in
\cite{allen83}. It takes as input some collection of intervals and their known relations, and
attempts to infer as many of the missing relations as possible. As mentioned
before, time intervals are primitives in Allen's interval algebra and since every two intervals
must be related by one of the symbols on table \ref{tab:basic_relations}, a directed graph with
labeled edges is a good format to store this information. In this representation the nodes of the
graph will be the intervals in question, and the labels on each edge will be sets of symbols that
describe the possible relations between the source and target intervals. As an example, suppose
there are two nodes $I$ and $J$ with the edge from $I$ to $J$ labelled by "< m o": this
means that one of $I < J$, $I\text{ m }J$ and $I\text{ o }J$ is expected to hold, although it is
uncertain which one. This labelling takes advantage of the fact that all the basic relations
are mutually exclusive, so the implied disjunctions in the above notation cause no ambiguities.
Similarly, the fact that the relation symbols are exhaustive also has a consequence for this
representation: the interval graph should be a complete directed graph, but as the 13 basic
relations all have a dual, it is always possible to tell what the label from $I$ to $J$ should be
by reading the label from $J$ to $I$. As a result, with some care, one can use a normal graph
as opposed to a directed graph, allowing for some saved space.

It is helpful to first define a helper function for the main algorithm: when given two relation
symbols $r1$ and $r2$ linking three intervals, $I\ r1\ J$ and $J\ r2\ L$, it is important to tell
what symbols can relate $I$ and $L$. This is done by looking up the relevant entry $T(r1,r2)$ of
table \ref{tab:transitivity}. Next, given two arcs going from $I$ to $J$ and $J$ to $L$, each
labelled by $R1$ and $R2$, let $\textit{Constraints}(R1,R2)$ be the maximum set of relation
symbols which could relate $I$ and $L$. The pseudocode for computing this can be found in
algorithm \ref{alg:constraints}.

\begin{table}[htpb]
  \centering
  \begin{tabular}{|c||p{8mm}|p{8mm}|p{8mm}|p{8mm}|p{8mm}|p{8mm}|p{8mm}|p{8mm}|p{8mm}|p{8mm}|p{8mm}|p{8mm}|}
\hline
      & < & m & o & s & f & d & > & M & O & S & F & D \\ \hline\hline
    < & < & < & < & < & < m o s d & < m o s d & full & < m o s d & < m o s d & < & < & < \\ \hline
    m & < & < & < & m & o s d & o s d & > M O S D & f = F & o s d & m & < & < \\ \hline
    o & < & < & < m o & o & o s d & o s d & > M O S D & O S D & con & o F D & < m o & < m o F D \\ \hline
    s & < & < & < m o & s & d & d & > & M & f d O & s = S & < m o & < m o F D \\ \hline
    f & < & m & o s d & d & f & d & > & > & > M O & > M O & f = F & > M O S D \\ \hline
    d & < & < & < m o s d & d & d & d & > & > & d f > O M & d f > O M & < m o s d & full \\ \hline
    > & full & d f > O M & d f > O M & d f > O M & > & d f > O M & > & > & > & > & > & > \\ \hline
    M & < m o F D & s = S & f d O & f d O & M & f d O & > & > & > & > & M & > \\ \hline
    O & < m o F D & o F D & con & f d O & O & f d O & > & > & > M O & > M O & O S D & > M O S D \\ \hline
    S & < m o F D & o F D & o F D & s = S & O & f d O & > & M & O & S & D & D \\ \hline
    F & < & m & o & o & f = F & o s d & > M O S D & O S D & O S D & D & F & D \\ \hline
    D & < m o F D & o F D & o F D & o F D & O S D & con & > M O S D & O S D & O S D & D & D & D \\ \hline
\end{tabular}

  \caption{Transitivity table for basic relations -- adapted from \cite{thomaswebsite}.\\
  Given relations $I\ r1\ J$ and $J\ r2\ L$,
  the possible relations between $I$ and $L$ will be under row $r1$ and column $r2$. Entries
  with the word "full" should contain all relations and entries labelled "con" should contain
  "o s f d = O S F D".}
  \label{tab:transitivity}
\end{table}

\begin{algorithm}
  \caption{Computing constraints given two edge labels. \cite{allen83}}\label{alg:constraints}
  \begin{algorithmic}
    \Function{Constraints}{R1, R2}
    \State $C \gets \emptyset$
    \ForAll{$r1 \in R1$}
      \ForAll{$r2 \in R2$}
        \State $C \gets C \cup T(r1,r2)$
      \EndFor
    \EndFor
    \State \Return C
    \EndFunction
  \end{algorithmic}
\end{algorithm}

Finally, the pseudocode to update a temporal network with a new label for a specific edge can be
seen in algorithm \ref{alg:main_algo}. It is assumed there exists a \textit{ToDo} queue,
where edges whose constraints need to be updated are put. Furthermore, for
intervals $i$ and $j$, $N(i,j)$ denotes the relations on the arc from $i$ to $j$ in the interval
graph and $R(i,j)$ denotes the new relations on the arc from $i$ to $j$. Lastly, in simpler cases
the \textit{Comparable} function can be taken to always return true, but as the number of
intervals grows, it is helpful to reduce the amount of updates needed through some optimisations
explained in chapter 5 of \cite{allen83}.

\begin{algorithm}
  \caption{Updating temporal network. \cite{allen83}}\label{alg:main_algo}
  \begin{algorithmic}
    \Procedure{To Add}{R(i, j)}
      \State Add $(i,j)$ to queue \textit{ToDo}
      \While{\textit{ToDo} is not empty}
        \State Get next $(i,j)$ from queue \textit{ToDo}
        \State $N(i,j) \gets R(i,j)$
        \ForAll{nodes $k$ such that \Call{Comparable}{k, j}}
          \State $R(k, j) \gets N(k,j) \cap \Call{Constraints}{N(k,j), R(i,j)}$
          \If{$R(k,j) \subset N(k,j)$} %TODO check if this should be j or i
            \State Add $(k,j)$ to \textit{ToDo}
          \EndIf
        \EndFor
        \ForAll{nodes $k$ such that \Call{Comparable}{i, k}}
          \State $R(i, k) \gets N(i,k) \cap \Call{Constraints}{R(i,j), N(j,k)}$
          \If{$R(i,k) \subset N(i,k)$}
            \State Add $(i,k)$ to \textit{ToDo}
          \EndIf
        \EndFor
      \EndWhile
    \EndProcedure
  \end{algorithmic}
\end{algorithm}

An interesting question of correctness comes for this algorithm. Is it possible to infer an
erroneous labelling of edges from a satisfiable starting graph? Thankfully this is not the case,
but this is not to say the algorithm can detect all invalid arrangements it is given. During the
execution of the algorithm, if at any point a pair of intervals $(k,l)$ is found such
that $R(k,l) = \empty$, then the given configuration of intervals is impossible to satisfy.
Unfortunately, as the labellings of edges are only updated by looking at paths of length two,
this will only find inconsistencies in three node subgraphs, so global inconsistencies might
not be noticed. If in doubt though, after the edges have all been updated, it is possible to
do a search for valid assignments, which will hopefully be easier than at the start.

\newpage

\subsection{Model Theory}%
\label{sub:model_theory}

We assume some familiarity with the basics of model theory, for which we recommend \cite{marker02}.
In this section we will focus mainly on the definitions needed to understand the work done in
\cref{sec:model_theory}.

First we introduce the idea of homogeneous structures, and how to construct examples of these
using Fraïssé limits. These structures will be characterised by their ability to extend isomorphisms
between substructures to automorphisms. As a result, homgeneous structures can have very interesting
automorphism groups and their study helps link model theory, group theory and combinatorics. For a
thorough survey of the area see \cite{macpherson11}.

Then we will cover some important classes of theories arising from classification theory, namely we
will introduce stable theories and one of their possible generalisations, NIP theories. The advent
of classification theory came with Morley's Categoricity Theorem, which related the number of
non-isomorphic models of different cardinalities for countable complete theories, often called the
spectrum of the theory. The concept of stable theories arose to study these spectra, offering a
definitive answer through the use of tools like forking and dividing. NIP theories generalise the
class of stable theories to include important examples like linear orders and geometric examples
like algebraically closed valued fields \cite{acvf-NIP} or the real exponential field
\cite{steinhorn1999}.

\subsubsection{Homogenous structures and Fraïssé classes}%
\label{ssub:homogeneous_structures_and_fraisse_classes}

We work with the definition of a homogenous structure found in \cite{macpherson11} as we are
mainly interested in the study of Fraïssé limits.

\begin{defn}
  A homogenous structure is a countable, possibly finite, relational structure (with finite
  language $\lang$) such that, for every isomorphism $f : U \to V$ between finite substructures $U$
  and $V$ of $M$, there is an authomorphism $f' : M \to M$ extending $f$.
\end{defn}

The simplest example of a homogeneous structure comes from the theory of strict linear orders, which
we now define.

\begin{defn}
  We define the language of strict linear orders $\lslo$ as the single binary relation $\{ < \}$.
\end{defn}

\begin{defn}
  We define the theory of strict linear orders as
  \begin{align*}
    \tslo = \big\{\; & \forall a, \lnot\ (a < a), \\
              & \forall a, \forall b, \forall c, (a < b) \land (b < c) \rightarrow (a < c) \\
              & \forall a, \forall b, (a < b) \lor (a = b) \lor (b < a) \;\big\}
  \end{align*}
\end{defn}

So a strict linear order consists of a binary relation on a set, which is irreflexive, transitive
satisfies the trichotomy condition. Notice that from these we can infer that the binary relation
must be antisymmetric, since if $(a < b) \land (b < a)$ then by transitivity $a < a$, which is
a contradiction with the irreflexivity of $<$.

\begin{prop}
  The strict linear order $(\Q,<)$ is homogeneous.
\end{prop}
\begin{proof}
  First we see how to expand the
  domain of an order isomorphism $f : L \to P$ with $L, P \subseteq \Q$ both finite. Suppose that we
  wish to extend the domain of $f$ to include some $a \in \Q \setminus L$. There are three cases to
  worry about here:
  \begin{itemize}
    \item If $a$ is an upper bound for $L$, we find some upper bound $b$ of $P$ which is
      not in $P$, such a $b$ must exist as $\Q$ is unbounded and $P$ is finite. Then we extend
      $f : L \to P$ to $g : L \cup \{a\} \to P \cup \{b\}$ by sending $g(a) = b$. This remains an
      order isomorphism since for any $x \in P$, we must have $x < a$ and $g(x) < b = g(a)$ since
      $a$ and $b$ are both upper bounds of $L$ and $P$ respectively.
    \item If $a$ is a lower bound for $L$, then we find a lower bound $b$ of $P$ not already in
      $P$ and extend $f$ to $g : L \cup \{a\} \to P \cup \{b\}$ by sending $a$ to $b$ again.
      Similarly to the upper bound case, since $a$ and $b$ are both lower bounds of the domain and
      codomain, this remains an order isomorphism.
    \item If $a$ is neither, then we notice that $L,P$ are finite linear orders and hence discrete.
      This means that we may find $a_1,a_2 \in L$ such that $a_1 < a < a_2$ and for no $x \in L$ do
      we have $a_1 < x < a_2$. Now we can find some $b \in \Q \setminus P$ such that
      $g(a_1) < b < g(a_2)$ since $\Q$ is dense and $P$ finite. We now extend $f$ to
      $g : L \cup \{a\} \to P \cup \{b\}$ by sending g(a) = b. This remains an order isomorphism
      since for any $x \in L$ we have either $x \leq a_1 < a$, so
      \begin{equation*}
        g(x) = f(x) \leq f(a_1) = g(a_1) < b = g(a)
      \end{equation*}
      or we have $a < a_2 \leq x$, in which case
      \begin{equation*}
        g(a) = b < g(a_2) = f(a_2) \leq f(x) = g(x)
      \end{equation*}
  \end{itemize}
  If we wanted to expand the codomain of $f$ in a similar way, we could just regard $f^{-1}$ as
  an order isomorphism between finite subsets of $\Q$, extend its domain to include whichever
  element we needed giving us a function $g$, then $g^{-1}$ would be the required extension of $f$.

  Now, fix two finite suborders $L,P \subseteq \Q$
  and suppose we have some order isomorphism $f : L \to P$. To extend $f$ to an automorphism of $\Q$
  we start by fixing an enumeration $(a_1, a_2, \dots)$ of the elements of $\Q$ and we define
  three sequences: $(L_1, L_2, \dots)$ and $(P_1, P_2, \dots)$ of increasing subsets of $\Q$ and
  $(g_1, g_2, \dots)$ of bijections $g_i : L_i \to P_i$ where each $g_i$ extends its predecessors.
  We define this sequences by induction:
  \begin{itemize}
    \item $k = 1$: let $L_1 = L,\ P_1 = P$ and $g_1 = f$.
    \item $k = 2l$ for $l \in \{1, 2, \dots\}$: at even indices we focus on increasing the domain of
      $g_i$ to all of $\Q$. If $a_l \in L_{k-1}$ then we let
      \begin{align*}
        L_k & = L_{k-1} \\
        P_k & = P_{k-1} \\
        g_k & = g_{k-1}
      \end{align*}
      Otherwise we extend $g_{k-1}$ to an order isomorphism
      $h : L_{k-1} \cup \{a_l\} \to P_{k-1} \cup \{b\}$ for some $b$ chosen appropriately and let
      \begin{align*}
        L_k & = L_{k-1} \cup \{a_l\} \\
        P_k & = P_{k-1} \cup \{b\} \\
        g_k & = h
      \end{align*}
    \item $k = 2l + 1$ for $l \in \{1, 2, \dots\}$: at odd indices we focus on increasing the
      codomain of $g_i$ to all of $\Q$. If $a_l \in P_{k-1}$ then we let
      \begin{align*}
        L_k & = L_{k-1} \\
        P_k & = P_{k-1} \\
        g_k & = g_{k-1}
      \end{align*}
      Otherwise we extend $g_{k-1}$ to an order isomorphism
      $h : L_{k-1} \cup \{b\} \to P_{k-1} \cup \{a_l\}$ for some $b$ chosen appropriately and let
      \begin{align*}
        L_k & = L_{k-1} \cup \{b\} \\
        P_k & = P_{k-1} \cup \{a_l\} \\
        g_k & = h
      \end{align*}
  \end{itemize}
  Finally, it is the case that $\Cup L_k = \Cup P_k = \Q$ since each $x \in \Q$ must appear as $a_l$
  in our enumeration for some $l \in \{1, 2, \dots\}$. Then $x \in L_{2l}$ and $x \in P_{2l+1}$ so
  definitely $x \in \Cup L_k$ and $x \in \Cup P_k$, so these unions must equal $\Q$. This means that
  we now have the function $g = \Cup g_k$ which extends $f$ by construction.
\end{proof}

The main bulk of the work above comes from the density of $\Q$, in fact, non dense linear orders are
only homogeneous in the trivial case.

\begin{prop}
  A non-dense strict linear order $L$ with more than one element is not homogeneous.
\end{prop}
\begin{proof}
  If the linear order is $\{x < y\}$ then the map partial isomorphism
  $x \mapsto y$ cannot be extended further.

  If the linear order has more than 3 elements, we can find either find $x < y < z$ or
  $z < x < y$ where $x$ and $y$ have no elements inbetween (as the order is not dense). In the first
  case, the partial order isomorphism
  \begin{equation*}
    f : \{x,z\} \to \{x,y\} \text{ sending } x \mapsto x \text{ and } z \mapsto y.
  \end{equation*}
  cannot be extended since there is no $w$ such that $f(x) = x < w < y = f(z)$, so $y$ cannot be
  added to the domain. By a similar argument, the following map handles the second case
  \begin{equation*}
    f : \{y,z\} \to \{x,y\} \text{ sending } y \mapsto y \text{ and } z \mapsto x.
  \end{equation*}
\end{proof}

\begin{cor}
  $\N$ and $\Z$ are not homogeneous.
\end{cor}


% Given a relational structure $M$ with language $\lang$, we denote by $\age(M)$ the class of finite
% $\lang$-structures isomorphic to a finite substructure of $M$. The age of a homogeneous structure
% $M$ satisfies some properties allowing $M$ to be reconstructed, up to isomorphism, from
% $\age(M)$. This leads to the question of when a class of finite $\lang$-structures is the age of some
% homogeneous $\lang$-structure.

% Given a countable relational structure $M$, it can be seen that $\age(M)$ must satisfy the HP,
% JEP and be EC. The hereditary property is satisfied because any subset of a finite set must also
% be finite, hence the substructure of an element of $\age(M)$ must itself be in $\age(M)$. As for
% the JEP, notice that we are working with a purely relational language, hence given two finite
% $\lang$-substructures of $M$, their union must also be finite and an $\lang$-substructure of $M$.
% Finally, because $\lang$ is finite, there can only be a finite number of isomorphism classes of
% $\lang$-structures with size $n \in \N$. This means the total number of isomorphism classes in
% $\age(M)$ is a countably infinite union of finite sets, which can be at most countably infinite.

Now, it would be interesting to see how to build homogeneous models of a theory. Towards this goal,
we will consider the class of finite $\lang$-structures which embed into a relational structure $M$.
This will be called the age of the structure $M$, denoted $\age(M)$. In general, the age of any
countable relational structure will satisfy the following three properties.

\begin{defn}
  A class $\C$ has the hereditary property (HP) if it is closed under taking substructures, so if
  $A, B$ are $\lang$-structures, $A \in \C$ and $B \subseteq A$ then $B \in \C$ too.
\end{defn}

\begin{defn}
  A class $\C$ has the joint embedding property (JEP) if for any $A,B \in \C$, we can find a
  third $\lang$-structure $C \in \C$ such that $A$ and $B$ both embed into it (by maps respecting the
  language $\lang$).
\end{defn}

\begin{defn}
  A class $\C$ is essentially countable (EC) if, up to isomorphism, there are only countably many
  $\lang$-structures.
\end{defn}

Given a class $\C$ satisfying the above three properties, then one may construct a countable model
$M$ such that $\age(M) = \C$ by enumerating all the isomorphism classes in $\C$ and gluing all of
these in order using the JEP. Suppose that $M$ is homogeneous though, then we can say something
further about its age, namely it will satisfy the following.

\begin{defn}
  A class $\C$ has the amalgamation property (AP) if for any span
  $A \longleftarrow C \longrightarrow B$ with $A,B,C \in \C$, there exists some $\lang$-structure
  $\Omega \in C$ with embeddings $A \longrightarrow \Omega \longleftarrow B$ making the
  following diagram commute.
  \[\begin{tikzpicture}
    \node[minimum size=7mm] (C)              {$C$};
    \node[minimum size=7mm] (B) [right=of C] {$B$};
    \node[minimum size=7mm] (A) [below=of C] {$A$};
    \node[minimum size=7mm] (O) [right=of A] {$\Omega$};
    \draw[->] (C.east) -- (B.west);
    \draw[->] (C.south) -- (A.north);
    \draw[->] (B.south) -- (O.north);
    \draw[->] (A.east) -- (O.west);
  \end{tikzpicture}\]
\end{defn}

Considering again a class of models $\C$ satisfying the HP, EC, and JEP, suppose that we assume that
$\C$ also has the AP. Then a theorem of Fraïssé tells us that gluing all of the models in $\C$
yields a homogeneous model $M$ with age $\C$.

\begin{thm}[Fraïssé's theorem]
  Given a class $\C$ of finite $\lang$-structures satisfying the properties HP, JEP, AP and EC, then
  there exists some homogeneous $\lang$-structure $M$ such that $\age(M) = \C$. Furthermore, if we
  have two such homogeneous $\lang$-structures $M$ and $N$, then necessarily $M \cong N$.
\end{thm}

A proof of this, along with proofs that $\age(M)$ satisfies the relevant properties can be found in
\cite{hodges93}. When the above structure $M$ exists for a class $\C$, then we call $M$ the Fraïssé
limit of $\C$.

Note that the uniqueness condition of the Fraïssé limit only applies to homogeneous
structures. In fact, if we consider the class of finite strict linear orders, we can see it
coincides with $\age(\Q)$, $\age(\Z)$ and also $\age(\N)$ despite neither of these linear orders
being isomorphic. This happens as neither $\Z$ nor $\N$ are homogeneous,
as seen earlier.

As we have decided not to show a lot of proofs for this section, we will instead consider in
detail how to compute the Fraïssé limit of the class $\finslo$ of finite strict linear orders,
to hopefully ground all these definitions a bit better. First, to see that this limit exists, we
need to check that $\finslo$ satisfies the 4 properties needed to apply Fraïssé's theorem.

First we prove some general results about the HP and EC which we will use later.

\begin{prop}
  Given a relational, universal theory $\theory$, the class of finite models of $\theory$ satisfies
  the hereditary property.
\end{prop}
\begin{proof}
  Fix some finite model $M \models \theory$ and a substructure
  $N \subseteq M$. Since $M$ was finite, $N$ must also be finite. Now, suppose we have a universal
  sentence
  \begin{equation*}
    \phi = \forall x_1, \dots, \forall x_n, \psi(x_1,\dots,x_n)
  \end{equation*}
  where $\psi$ is a quantifier free formula such that $M \models \phi$. Fixing some tuple $(a_1,\dots,a_n) \in N^n$ we
  know that $M \models \psi(a_1,\dots,a_n)$, but $N$ is a substructure of $M$ and the truth value
  of quantifier free formulas is preserved by embeddings, so $N \models \psi(a_1,\dots,a_n)$ too.
  As we fixed an arbitrary tuple of elements in $N$, this means that $N \models \phi$.
  The theory $\theory$ is taken to be universal, hence all sentences $\phi \in \theory$ are
  equivalent to some universal sentence modulo $\theory$. As said universal sentence is preserved
  under taking substructures, so must be $\phi$. Hence if $M \models \theory$, then any substructure
  $N \subseteq M$ will also be a model of $\theory$.
\end{proof}

\begin{cor}
  The class $\finslo$ has the HP.
\end{cor}

\begin{prop}
  For any finite language $\lang$, the class of all finite $\lang$-structures is essentially
  countable.
\end{prop}
\begin{proof}
  Fix some finite language
  \begin{equation*}
    \lang = \{c_1, \dots, c_m, f_1, \dots, f_n, R_1, \dots, R_l \}
  \end{equation*}
  where the $c_i$ are constant symbols, the $f_i$ are function symbols and the $R_i$ are relation
  symbols. We will show that the class of finite $\lang$-structures must be essentially countable.
  First, consider the set $M = \{1,\dots,r\} \subset \N$. If we wish to turn $M$ into a
  $\lang$-structure then we must pick interpretations for all the symbols in $\lang$. For a constant
  symbol this consists of picking a single element, so there are $r$ possible options. For a
  function symbol with $a$-arity, we need to pick an element of $M$ for every input tuple of $a$
  elements of $M$, so there are $r^{r^a}$ possibilities. Similarly, for a relation symbol $R$ with
  $b$-arity, we need to pick a truth value for all tuples of size $b$ in $M$, so there are $2^{r^b}$
  options. Hence, if we denote by $a_i$ the arity of the symbol $f_i$ and $b_i$ the arity of the
  symbol $R_i$, then the total number of possible $\lang$-structures on $M$ is finite, more
  specifically it is
  $r^{\left(m + \sum_{i=1}^{n} r^{a_i}\right)}2^{\left(\sum_{i=1}^{l} r^{b_i}\right)}$.
  As such, the set of
  possible $\lang$-structures on all initial segments $\{1,\dots,r\} \subseteq \N$ is countable.
  Every finite $\lang$-structure must be isomorphic to at least one of these, so the class of finite
  $\lang$-structures is essentially countable.
\end{proof}

\begin{cor}
  The class $\finslo$ is EC.
\end{cor}
\begin{proof}
  The class $\finslo$ is a subclass of all finite strict linear orders, hence if the latter is EC,
  then the first must also be EC.
\end{proof}

Now we focus solely on strict linear orders.

\begin{prop}
  The class $\finslo$ has the joint embedding property.
\end{prop}
\begin{proof}
  Given two strict linear orders $L,P$, we can turn their disjoint union $L \sqcup P$ into a
  strict linear order by carrying over the orderings from $L$ and $P$ and setting
  $x < y$ for all $x \in L$ and $y \in P$. Then, the usual injections into the disjoint union
  $f : L \to L \sqcup P$ and $g : P \to L \sqcup P$ become order preserving maps. In the case
  that $L,P$ are both finite, so will $L \sqcup P$ be, hence the class of finite strict
  linear orders has the joint embedding property.
\end{proof}

\begin{prop}
  The class $\finslo$ has the amalgamation property.
\end{prop}
\begin{proof}
  Given linear orders $A,L,P$ and order embeddings $f : A \to L$, $g : A \to P$, we wish to
  find a linear order $\Omega$ and order embeddings $f' : L \to \Omega$, $g' : P \to \Omega$ such
  that $f' \circ f = g' \circ g$. If any of $A,L,P$ is empty, we devolve to the previous proof,
  so we may assume that all of $A,L,P$ are non-empty. We take the product $L \times P$ and
  order it lexicographically, such that for all $(x,y), (x',y') \in L \times P$
  \begin{equation*}
    (x,y) < (x',y') \iff (x < y') \lor ((x = x') \land (y < y'))
  \end{equation*}
  Fixing some $y \in P$ we define $f' : L \to L \times P$ by
  \begin{equation*}
    f'(x) = \begin{cases}
      (x, g(a)) & \text{ if} f^{-1}(x) = \{a\} \\
      (x, y)    & \text{ if } f^{-1}(x) = \emptyset
    \end{cases}
  \end{equation*}
  This preserves the ordering of $L$ as it coincides with the identity map on the first coordinate
  and we are ordering $L \times P$ lexicographically.
  Defining $g' : P \to L \times P$ is slightly trickier. First, we must pick some $y_{L,U} \in L$
  such that $L \leq y_{L,U} \leq U$ for all subsets $L,U \subseteq L$ such that $L < U$. Once
  we have fixed our choices, for any $x \in P$ we denote
  $y_x =  y_{\left\{f(a)\ |\ g(a) < x\right\}, \left\{f(a)\ |\ x < g(a)\right\}}$.
  Finally, we define
  \begin{equation*}
    g'(x) = \begin{cases}
      (f(a), x) & \text{ if} g^{-1}(x) = \{a\} \\
      (y_x, x)  & \text{ if } f^{-1}(x) = \emptyset
    \end{cases}
  \end{equation*}
  To check this preserves the ordering of $P$, we fix two elements $x < x'$ of $P$, then:
  \begin{itemize}
    \item If $g^{-1}(x) = \{a\}$ and $g^{-1}(x') = \{b\}$, then $a < b$, which implies that
      $f(a) < f(b)$, so $g'(x) = (f(a),x) < (f(b),x') = g'(x')$.
    \item If $g^{-1}(x) = \{a\}$ and $g^{-1}(x') = \emptyset$, then we have picked $y_x$ such that
      $f(a) < y_x$ so $g'(x) = (f(a),x) < (y_x, x) = g'(x')$.
    \item If $g^{-1}(x) = \emptyset$ and $g^{-1}(x') = \{a\}$, then we have picked $y_x$ such that
      $y_x < f(a)$ so $g'(x) = (y_x, x) < (f(a),x) = g'(x')$.
    \item If $g^{-1}(x) = g^{-1}(x') = \emptyset$ then notice that $y_x \leq y_{x'}$ since
      either there exists some $a \in A$ such that $x < g(a) < x'$, so then
      $y_x \leq g(a) \leq y_{x'}$ or no such $a$ exists and $y_x = y_{x'}$. Thus,
      $g'(x) = (y_x,x) < (y_{x'}, x') = g'(x')$.
  \end{itemize}
  Now that we have the required order embeddings into $\Omega = L \times P$, we just check the
  necessary commutativity condition:
  \begin{equation*}
    f' \circ f(a) = f'(f(a)) = (f(a),g(a)) = g'(g(a)) = g' \circ g(a)
  \end{equation*}
  In the case that $A,L,P$ are finite, then $L \times P$ will also be finite, so this
  shows that the class of finite strict linear orders has the amalgamation property.
\end{proof}

\begin{thm}
  The Fraïssé limit of $\finslo$ is $\Q$ with its usual order.
\end{thm}
\begin{proof}
  As $\finslo$ satisfies the HP, JEP, AP and is EC, then it must have a Fraïssé limit $M$. We saw
  before that $\Q$ is homogeneous. By uniqueness of the Fraïssé limit, it suffices to show that
  $\age(\Q) = \finslo$. Clearly $\age(\Q) \subseteq \finslo$ since
  a suborder of a linear order must still be linear. To see that $\finslo \subseteq \age(\Q)$ we fix
  some finite linear order $L$, then there is an order preserving isomorphism from $L$ to an initial
  segment of $\N$. The inclusion $\N \in \Q$ means this isomorphism realises $L$ as a finite
  suborder of $\Q$.
\end{proof}

\subsubsection{Stable theories}%
\label{ssub:stable_theories}

\begin{defn}
  Let $T$ be a complete theory. For an infinite cardinal $\kappa$, $T$ is
  $\kappa$-stable if for every every model $M$ of $T$ and subset $A \subseteq M$ with cardinality
  $|A| = \kappa$, then the set of complete $n$-types in $M$ over $A$,  $S_n^M(A)$, has cardinality
  $\kappa$.
\end{defn}

If a theory $T$ is not $\kappa$-stable for any infinite cardinal $\kappa$, then it is called
unstable. Given a model $M$, we say that $M$ is stable (resp. unstable) if the full theory of $M$
is stable (resp. unstable).

The following theorem gives us a characterisation of stability in terms of linear orders, which can
be simpler to reason with, especially when one considers the close relation between linear orders
and interval algebras.

\begin{thm}\label{thm:stable_order_property}
  Let $\theory$ be a complete theory, then $\theory$ is stable if and only
  if there exists a formula $\phi(v_1,\dots,v_n;w_1,\dots,w_n)$ and a model $M \models \theory$ with
  a sequence $x_1,x_2,\dots \in M^n$ such that
  \begin{equation*}
    M \models \phi(x_i;x_j) \iff i < j
  \end{equation*}
  Such a formula is said to have the order property.
\end{thm}

The proof for this is somewhat involved, so we refer the interested reader to \cite{marker02}.

\begin{cor}
  A linear order $L$ is stable if and only if it is finite.
\end{cor}

\begin{exmp}
  A complete theory is strongly minimal if for all models $M$, any definable set (with parameters)
  $D \subseteq M$ is either finite or cofinite. This turns out to be a very strong requirement,
  meaning that strongly minimal theories must also be stable. The proof for this relies on the
  equivalence between stability of a theory and non-existence of a formula with the strict order
  property (which is slightly different from the order property), and so is omitted.

  We know from \cite{marker02} that the following are strongly minimal and hence stable:
  \begin{itemize}
    \item The theory of algebraically closed fields in characteristic $p$ in the language of rings:
      all definable sets of an algebraically closed field $k$ are boolean combinations of zero sets
      of polynomials in $k[x]$. Since these zero sets are either finite or all of $k$, then the
      claim of strong minimality follows.
    \item The theory of $\Q$-vector spaces in the language of modules: for a $\Q$-vector space $V$,
      all definable sets $D \subseteq V$ are given by boolean combinations of formulas of the form
      $nx = a$ where $n \in \N$, and $a \in V$. If $a$ is nonzero, such a formula can have at most
      one solution, hence $D$ will have to be either finite or cofinite.
  \end{itemize}
\end{exmp}

\subsubsection{NIP theories}
\label{ssub:nip_theories}

\begin{defn}
  For a complete theory $\theory$, we say that a formula $\phi(x;y)$ has the independence property
  if there is a model $M \models \theory$ and sequences $(a_i)_{i<\omega}$,
  $(b_I)_{I \subseteq \omega}$ in $M$ such that
  \begin{equation*}
    M \models \phi(a_i,b_I) \iff i \in I
  \end{equation*}
  If a formula does not have the IP, we say it has the NIP.
\end{defn}

\begin{defn}
  A complete theory has the IP if there exists some formula with the IP. It has the NIP if all
  formulas have the NIP.
\end{defn}

By compactness, to show that a specific formula has the IP, it suffices to consider only arbitrarily
large finite sequences.

\begin{prop}
  For a complete theory $\theory$, a formula $\phi(x;y)$ has the IP if and only if the following
  is satisfiable
  \begin{equation*}
    \theory \cup \{\phi(a_i,b_I)\ |\ i \in I \subseteq \{0,1,\dots,n\}\}
  \end{equation*}
  for arbitrarily large $n$ (where the $a_i$ and $b_I$ are new constant symbols).
\end{prop}
\begin{proof}
  The forwards implication follows by taking the model and sequences $(a_i)_{i<\omega}$,
  $(b_I)_{I \subseteq \omega}$ which realise the IP for $\phi(x;y)$ and discarding the
  $a_i$ with $i > n$ and $I \nsubseteq \{0,1,\dots,n\}$.

  The converse, showing that $\phi(x;y)$ has the IP amounts to showing
  the satisfiability of
  \begin{equation*}
    \theory \cup \{\phi(a_i,b_I)\ |\ i \in I \subseteq \N \}
  \end{equation*}
  But by compactness, it suffices to show satisfiability of
  \begin{equation*}
    \theory \cup \{\phi(a_{i_1},b_{I_1}), \phi(a_{i_2},b_{I_2}) \dots, \phi(a_{i_m},b_{I_m})\}
  \end{equation*}
  where $i_k \in I_k$ for all $k$. Letting $n = \max(i_1,i_2,\dots,i_m)$ and applying our
  hypothesis, we see this is indeed satisfiable.
\end{proof}

This means that if a theory has the NIP, then for all formulas $\phi(x;y)$ and all models $M$,
there exists a maximum $n \in \N$ such that for all $\{a_1, \dots, a_n\} \subseteq M$, there exists
some subset ${a_{i_1}, \dots, a_{i_m}}$ which we cannot pick out with $\phi$ by varying the
parameter $y$.

Now, it would be helpful to see both some examples and non-examples of NIP theories. The
following proposition helps with the former.

\begin{prop}
  All stable theories have the NIP.
\end{prop}
\begin{proof}
  Suppose we have a theory $\theory$ which has the IP, so the IP is realised for some formula
  $\phi(x;y)$, a model $M \models \theory$ and elements $(a_i)_{i<\omega}$,
  $(b_I)_{I \subseteq \omega}$. Then the formula $\psi(x,x';y,y') = \phi(x;y')$ is unstable,
  since if we define the sequence $(c_i)_{i<\omega}$ by $c_i=(a_i,b_{\{n | n < i\}})$ then
  \begin{equation*}
    M \models \psi(c_i;c_j) \iff
      M \models \phi(a_i; b_{\{n|n < j\}}) \iff
      i \in \{n\ |\ n < j\} \iff
      i < j
  \end{equation*}
  This shows that $\psi$ has the order property, so $\theory$ must be unstable.
\end{proof}

This means that all our examples of stable theories are also examples of NIP theories. However not
all NIP theories are stable, for example all linear orders, infinite or not, are NIP. The proof of
this requires some machinery which is not relevant to the rest of this paper, but the details
can be found in \cite{simon15}.


\begin{exmp}
  The real exponential field $(\R,+,\cdot,0,1,e^x)$ has the NIP \cite{steinhorn1999}. This means
  that for any formula $\phi(x;y)$ there exists some maximum $n$ such that
  \begin{equation*}
    \fulltheory(\R) \cup \{\phi(a_i,b_I)\ |\ i \in I \subseteq \{0,1,\dots,n\}\}
  \end{equation*}
  is satisfiable. We refer to this $n$ as the VC dimension of $\phi(x;y)$. This notion of VC
  dimension is not only relevant in model theory, in machine learning it can be used to measure the
  expressive power of a classification model. One interesting and often studied class of
  classification models comes from feedforward neural networks with sigmoid activation functions. It
  turns out that any such neural network can be expressed as a first order formula in the language
  of exponential fields with its weights as parameters \cite{macintyre93} and as such any
  feedforward neural network with sigmoid activation functions will have finite VC dimension.
\end{exmp}

Interestingly, by modifying the above slightly to consider the complex exponential field, we get a
theory with the IP, which in a sense, indicates that having an NIP theory is quite a special
occurrence, and even slight variations may give a theory the IP.

\begin{exmp}
  To see why the complex exponential field $(\mathbb{C},+,\cdot,0,1,e^x)$ has the IP, we will show
  that we can define the set of integers. For this, notice that the set $\{i,-i\}$ is defined by
  the formula $\phi(x) = x \cdot x = 1$. Then, recall that we have
  \begin{equation*}
    \sin(\theta) = \frac{e^{i\theta} - e^{(-i)\theta}}{2i}
                 = \frac{e^{(-i)\theta} - e^{i\theta}}{2(-i)}
  \end{equation*}
  hence we can define sine using the formula
  \begin{equation*}
    \psi(x,y) = \exists a, \exists b, (a \neq b) \land \phi(a) \land \phi(b) \land
      \left(y = \frac{e^{ax} - e^{bx}}{2a}\right)
  \end{equation*}
  In turn, this allows us to define $2\pi \Z \subseteq \mathbb{\C}$ as the zero set of the sine
  function. Finally, the following formula, which essentially says that as integers, $y$ divides
  $x$, has the IP
  \begin{equation*}
    \varphi(x;y,\pi) = \exists z, \psi(z,0) \land \left(y \cdot \frac{z}{2\pi} = x\right)
  \end{equation*}
  which we show by considering arbitrarily large finite sets. Fix some $n$, then the subsets of
  $\{1,\dots,n\}$ can be put in bijection with $\{1,\dots,n^2\}$ under some $f$. Using $p_k$
  to refer to the $k$th prime, we let
  \begin{equation*}
    b_I = (p_{f(I)}, \pi)
  \end{equation*}
  for each $I \subseteq \{1,\dots,n\}$. Also for each $i \in \{1,\dots,n\}$ we let
  \begin{equation*}
    a_i = \prod_{\substack{J \subseteq \{1,\dots,n\},\\ i \in J}} p_{f(J)}
  \end{equation*}
  With the use of these parameters
  \begin{equation*}
    \C \models \varphi(a_i;b_I)
      \iff p_{f(I)}\ | \prod_{\substack{J \subseteq \{1,\dots,n\},\\ i \in J}} p_{f(J)}
      \iff i \in I
  \end{equation*}
  As this works for arbitrarily $n$, $\varphi(x;y,a)$ must have the IP.
\end{exmp}
