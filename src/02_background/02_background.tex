\documentclass[../main.tex]{subfiles}

\begin{document}

\section{Background}%
\label{sec:background}

\subsection{Allen's Interval Algebra}%
\label{sub:allen_interval_algebras}

The concept of Allen's interval algebras was first introduced in \cite{allen83}. The main
idea was to introduce a new system for arguing about time intervals in a qualitative way, similar
to how humans think about time. When one relates to someone else how some events might have
happened, it is rarely done with specific figures in mind. For example, when retelling a story
people will convey some ordering of events' start and end times, but skipping over the details
of the figures. Now, this might happen because the specific time frames don't add to the
story, or even because they are not known. Regardless of the reason, it is often helpful to argue
about time without reference to specific times. Computers however, tend to argue about time in a
quantitative manner: saving timestamps along with logs, such that if something goes wrong, the
order of events can be gotten; checking the system clock to tell if access tokens have expired;
so on. The difficulties of expressing time qualitatively on a computer become even more obvious
when working towards artificial intelligence, like in natural language processing, where the
computer must understand the timing of events from people's informal speech or writing.

Interval algebras were Allen's approach to have computers reasong about time as humans did and
still do. This was done by treating time intervals as a primitive notion, with binary relations
recording their ordering and level of overlap. Allen's interval algebra consists of 13 basic
relations which allow one to describe fully how any two intervals $I$ and $J$ might relate. The
relations and their meaning can be found on table \ref{tab:basic_relations}: relations come in
pairs (apart from $=$) since for each relation, one can get its dual (or opposite) relation.
If it is known that \(I < J\), then it can be immediately infered that \(J > I\). Due to this
duality, the meaning of the opposite relations has been omitted to save space, although it can be
got by swapping the roles of $I$ and $J$. We have also used $I-$ and $I+$ as shorthands for the
start and end points of the interval $I$.

\begin{table}[htpb]
  \centering
  \begin{tabular}{|l|c|c|c|}
    \hline
    Relation & Symbol & Dual Symbol & Meaning \\
    \hline
    \(I\) starts \(J\)
             & \(I < J\)
             & \(I > J\)
             & \(I- < I+ < J- < J+\)\\
    \(I\) meets \(J\)
             & \(I\text{ m }J\)
             & \(I\text{ M }J\)
             & \(I- < I+ = J - < J+\)\\
    \(I\) overlaps \(J\)
             & \(I\text{ o }J\)
             & \(I\text{ O }J\)
             & \(I- < J- < I+ < J+\)\\
    \(I\) starts \(J\)
             & \(I\text{ s }J\)
             & \(I\text{ S }J\)
             & \(J- < I- < I+ < J+\)\\
    \(I\) finishes \(J\)
             & \(I\text{ f }J\)
             & \(I\text{ F }J\)
             & \(I- = J- < I+ < J+\)\\
    \(I\) during \(J\)
             & \(I\text{ d }J\)
             & \(I\text{ D }J\)
             & \(J- < I- < I+ = J+\)\\
    \(I\) equals \(J\)
             & \(I = J\)
             & \(I = J\)
             & \(I- = J- < I+ = J+\)\\
    \hline
  \end{tabular}
  \caption{13 Basic Relations of Allen's Interval Algebra}
  \label{tab:basic_relations}
\end{table}

The choice of these 13 relations has some advantages which simplify reasoning: this set of
relations is both exhaustive and mutually-exclusive. The first means that for any two intervals
$I$ and $J$, there exists at least one relation R  such that $I\text{ R }J$. The latter
means there exists at most one such relation. These facts will be important when coming up
with a first-order axiomatisation of Allen's interval algebras, so it will be helpful to keep them
in mind going forward.

Reading the "Meaning" column of table \ref{tab:basic_relations}, it seems to be the case that
no time points are not considered as valid intervals, since $I-$ is always strictly less than
$I+$. This is no accident, as time points can cause ambiguities and thus complicate the
semantics of interval algebras. This disallowance can be problematic too though, since the
concept of points in time is often used without thought in natural language. For example, the 
phrase "I caught the ball" suggests that the act of catching the ball was instantaneous, and in a
sense it was, at least compared to the timescales used in everyday language. On the other hand,
the sentence "I met with them yesterday" clearly gives an interval of 24 hours when this meeting
might have taken place. In reality, what is or is not a time point depends a lot on the
context of the sentence: in a history lesson, it is entirely reasonable to think of days as
points, but describe to someone the big bang and nanoseconds suddenly become long intervals of
time. This reliance on context should evidence that interval algebras do not need the concept
of time points and that these can be modelled by suitably intervals.

\subsubsection{Basic Algorithm}%
\label{ssub:basic_algorithm}

With some of the main ideas explained, now is a good time to see the basic algorithm described in
\cite{allen83}. It takes as input some collection of intervals and their known relations, and
attempts to infer as many of the missing relations as possible. As mentioned
before, time intervals are primitives in Allen's interval algebra and since every two intervals
must be related by one of the symbols on table \ref{tab:basic_relations}, a directed graphi with
labeled edges is a good format to store this information. In this representation the nodes of the
graph will be the intervals in question, and the labels on each edge will be sets of symbols that
describe the possible relations between the source and target intervals. As an example, suppose
there are two nodes $I$ and $J$ with the edge from $I$ to $J$ labelled by "< m o": this
means that one of $I < J$, $I\text{ m }J$ and $I\text{ o }J$ is expected to hold, although it is
uncertain which one. This labelling takes advantage of the fact that all the basic relations
are mutually exclusive, so the implied disjunctions in the above notation cause no ambiguities.
Similarly, the fact that the relation symbols are exhaustive also has a consequence for this
representation: the interval graph should be a complete directed graph, but as the 13 basic
relations all have a dual, it is always possible to tell what the label from $I$ to $J$ should be
by reading the label from $J$ to $I$. As a result, with some care, one can use a normal graph
as opposed to a directed graph, allowing for some saved space.

It is helpful to first define a helper function for the main algorithm: when given two relation
symbols $r1$ and $r2$ linking three intervals, $I\ r1\ J$ and $J\ r2\ L$, it is important to tell
what symbols can relate $I$ and $L$. This is done by looking up the relevant entry $T(r1,r2)$ of
table \ref{tab:transitivity}. Next, given two arcs going from $I$ to $J$ and $J$ to $L$, each 
labelled by $R1$ and $R2$, let $\textit{Constraints}(R1,R2)$ be the maximum set of relation
symbols which could relate $I$ and $L$. The pseudocode for computing this can be found in
algorithm \ref{alg:constraints}.

\begin{table}[htpb]
  \centering
  \begin{tabular}{|c||p{8mm}|p{8mm}|p{8mm}|p{8mm}|p{8mm}|p{8mm}|p{8mm}|p{8mm}|p{8mm}|p{8mm}|p{8mm}|p{8mm}|}
\hline
      & < & m & o & s & f & d & > & M & O & S & F & D \\ \hline\hline
    < & < & < & < & < & < m o s d & < m o s d & full & < m o s d & < m o s d & < & < & < \\ \hline
    m & < & < & < & m & o s d & o s d & > M O S D & f = F & o s d & m & < & < \\ \hline
    o & < & < & < m o & o & o s d & o s d & > M O S D & O S D & con & o F D & < m o & < m o F D \\ \hline
    s & < & < & < m o & s & d & d & > & M & f d O & s = S & < m o & < m o F D \\ \hline
    f & < & m & o s d & d & f & d & > & > & > M O & > M O & f = F & > M O S D \\ \hline
    d & < & < & < m o s d & d & d & d & > & > & d f > O M & d f > O M & < m o s d & full \\ \hline
    > & full & d f > O M & d f > O M & d f > O M & > & d f > O M & > & > & > & > & > & > \\ \hline
    M & < m o F D & s = S & f d O & f d O & M & f d O & > & > & > & > & M & > \\ \hline
    O & < m o F D & o F D & con & f d O & O & f d O & > & > & > M O & > M O & O S D & > M O S D \\ \hline
    S & < m o F D & o F D & o F D & s = S & O & f d O & > & M & O & S & D & D \\ \hline
    F & < & m & o & o & f = F & o s d & > M O S D & O S D & O S D & D & F & D \\ \hline
    D & < m o F D & o F D & o F D & o F D & O S D & con & > M O S D & O S D & O S D & D & D & D \\ \hline
\end{tabular}

  \caption{Transitivity table for basic relations -- adapted from \cite{thomaswebsite}.\\
  Given relations $I\ r1\ J$ and $J\ r2\ L$,
  the possible relations between $I$ and $L$ will be under row $r1$ and column $r2$. Entries
  with the word "full" should contain all relations and entries labelled "con" should contain
  "o s f d = O S F D".}
  \label{tab:transitivity}
\end{table}

\begin{algorithm}
  \caption{Computing constraints given two edge labels. \cite{allen83}}\label{alg:constraints}
  \begin{algorithmic}
    \Function{Constraints}{R1, R2}
    \State $C \gets \emptyset$
    \ForAll{$r1 \in R1$}
      \ForAll{$r2 \in R2$}
        \State $C \gets C \cup T(r1,r2)$
      \EndFor
    \EndFor
    \State \Return C
    \EndFunction
  \end{algorithmic}
\end{algorithm}

Finally, the pseudocode to update a temporal network with a new label for a specific edge can be
seen in algorithm \ref{alg:main_algo}. It is assumed there exists a \textit{ToDo} queue,
where edges whose constraints need to be updated are put. Furthermore, for
intervals $i$ and $j$, $N(i,j)$ denotes the relations on the arc from $i$ to $j$ in the interval
graph and $R(i,j)$ denotes the new relations on the arc from $i$ to $j$. Lastly, in simpler cases
the \textit{Comparable} function can be taken to always return true, but as the number of
intervals grows, it is helpful to reduce the amount of updates needed through some optimisations
explained in chapter 5 of \cite{allen83}.

\begin{algorithm}
  \caption{Updating temporal network. \cite{allen83}}\label{alg:main_algo}
  \begin{algorithmic}
    \Procedure{To Add}{R(i, j)}
      \State Add $(i,j)$ to queue \textit{ToDo}
      \While{\textit{ToDo} is not empty}
        \State Get next $(i,j)$ from queue \textit{ToDo}
        \State $N(i,j) \gets R(i,j)$
        \ForAll{nodes $k$ such that \Call{Comparable}{k, j}}
          \State $R(k, j) \gets N(k,j) \cap \Call{Constraints}{N(k,j), R(i,j)}$
          \If{$R(k,j) \subset N(k,j)$} %TODO check if this should be j or i
            \State Add $(k,j)$ to \textit{ToDo}
          \EndIf
        \EndFor
        \ForAll{nodes $k$ such that \Call{Comparable}{i, k}}
          \State $R(i, k) \gets N(i,k) \cap \Call{Constraints}{R(i,j), N(j,k)}$
          \If{$R(i,k) \subset N(i,k)$}
            \State Add $(i,k)$ to \textit{ToDo}
          \EndIf
        \EndFor
      \EndWhile
    \EndProcedure
  \end{algorithmic}
\end{algorithm}

An interesting question of correctness comes for this algorithm. Is it possible to infer an
erroneous labelling of edges from a satisfiable starting graph? Thankfully this is not the case,
but this is not to say the algorithm can detect all invalid arrangements it is given. During the
execution of the algorithm, if at any point a pair of intervals $(k,l)$ is found such
that $R(k,l) = \empty$, then the given configuration of intervals is impossible to satisfy.
Unfortunately, as the labellings of edges are only updated by looking at paths of length two,
this will only find inconsistencies in three node subgraphs, so global inconsistencies might
not be noticed. If in doubt though, after the edges have all been updated, it is possible to
do a search for valid assignments, which will hopefully be easier than at the start.

\newpage

\subsection{Model Theory}%
\label{sub:model_theory}

For a proper introduction to model theory see \cite{marker02}, the focus of this section shall be
on concepts that might not be found in an introductory course to model theory.

\subsubsection{Homogenous structures and Fraïssé classes}%
\label{ssub:homogeneous_structures_and_fraisse_classes}

We work with the definition of a homogenous structure found in \cite{macpherson11} as we are
mainly interested in the study of Fraïssé limits.

\begin{defn}
  A homogenous structure is a countable, possibly finite, relational structure (with finite
  language $\lang$) such that, for every isomorphism $f : U \to V$ between finite substructures $U$
  and $V$ of $M$, there is an authomorphism $f' : M \to M$ extending $f$.
\end{defn}

\begin{exmp}
  The rational numbers viewed as a strict linear order are homogeneous. This can be seen by a
  standard back and forth argument to extend $f$ to an automorphism on all of $\Q$. Suppose
  we have an order preserving isomorphism $f : U \to V$ between finite substructures of $\Q$. If
  we fix any $a \in \Q \setminus U$, we can find the elements adjacent to $a$ inside of $U$,
  denote them $x < a < y$. Then, since $\Q$ is dense, we can find some element
  $b \in \Q \setminus V$ such that $f(x) < b < f(y)$. With this, we can extend $f$ to
  $f' : U \cup \{a\} \to V \cup \{b\}$ by setting $f'(a)=b$. The same argument works if
  $a$ only has one adjacent element in $U$, one just has to remove the irrelevant inequalities.
  By iterating this process to $f$ and its inverse, we can get the desired automorphism on all of
  $\Q$.
\end{exmp}

\begin{exmp}
  Non-dense linear orders with more than 2 elements are \textit{not} homogeneous. To see this,
  fix such a linear order and pick 2  elements $x,y$ with no elements in between. Since it there
  exist at least 3 elements, pick another $z$ in our linear order. If $x < y < z$, consider
  the map $f : \{x,z\} \to \{x,y\}$ sending $x \mapsto x, z \mapsto y$. Because no elements lie
  between $x$ and $y$, we cannot extend this map so that its domain is $\{x,y,z\}$. A similar
  map works in the case that $z < x < y$.
\end{exmp}

Given a homogenous structure $M$ with language $\lang$, we denote by $\age{M}$ the class of finite
$\lang$-structures isomorphic to a finite substructure of $M$. The age of a homogeneous structure
$M$ satisfies some properties which allow for $M$ to be reconstructed, up to isomorphism, from
$\age{M}$. This leads to the question of when a class of finite $\lang$-structures is the age of some
homogeneous $\lang$-structure.

\begin{defn}
  A class $\C$ has the hereditary property (HP) if it is closed under taking substructures, so if
  $A, B$ are $\lang$-structures, $A \in \C$ and $B \subseteq A$ then $B \in \C$ too.
\end{defn}

\begin{defn}
  A class $\C$ has the joint embedding property (JEP) if for any $A,B \in \C$, we can find a
  third $\lang$-structure $C \in \C$ such that $A$ and $B$ both embed into it (by maps respecting the
  language $\lang$).
\end{defn}

\begin{defn}
  A class $\C$ has the amalgamation property (AP) if for any span
  $A \longleftarrow C \longrightarrow B$ with $A,B,C \in \C$, there exists some $\lang$-structure
  $\Omega \in C$ with embeddings $A \longrightarrow \Omega \longleftarrow B$ making the
  following diagram commute.
  \[\begin{tikzpicture}
    \node[minimum size=7mm] (C)              {$C$};
    \node[minimum size=7mm] (B) [right=of C] {$B$};
    \node[minimum size=7mm] (A) [below=of C] {$A$};
    \node[minimum size=7mm] (O) [right=of A] {$\Omega$};
    \draw[->] (C.east) -- (B.west);
    \draw[->] (C.south) -- (A.north);
    \draw[->] (B.south) -- (O.north);
    \draw[->] (A.east) -- (O.west);
  \end{tikzpicture}\]
\end{defn}

%\todo{talk about fields and difference between JEP and AP}

\begin{defn}
  A class $\C$ is essentially countable (EC) if, up to isomorphism, there are only countably many
  $\lang$-structures.
\end{defn}

Given a countable relational structure $M$, it can be seen that $\age{M}$ must satisfy the HP,
JEP and be EC. The hereditary property is satisfied because any subset of a finite set must also
be finite, hence the substructure of an element of $\age{M}$ must itself be in $\age{M}$. As for
the JEP, notice that we are working with a purely relational language, hence given two finite
$\lang$-substructures of $M$, their union must also be finite and an $\lang$-substructure of $M$.
Finally, because $\lang$ is finite, there can only be a finite number of isomorphism classes of
$\lang$-structures with size $n \in \N$. This means the total number of isomorphism classes in
$\age{M}$ is a countably infinite union of finite sets, which can be at most countably infinite.

With the further constraint on $M$ to be a homogenous structure, it turns out that $\age{M}$ must
also have the AP. For a proof see \cite{hodges93}.

%\todo{prove that $\age{M}$ has the AP if $M$ is homogeneous}

Next, we have a theorem of Fraïssé, which tells us when a class $\C$ of finite $\lang$-structures is
the age of some homogeneous structure $M$.

\begin{thm}[Fraïssé's theorem]
  Given a class $\C$ of finite $\lang$-structures satisfying the properties HP, JEP, AP and EC, then
  there exists some homogeneous $\lang$-structure $M$ such that $\age{M} = \C$. Furthermore, if we
  have two such homogeneous $\lang$-structures $M$ and $N$, then necessarily $M \cong N$.
\end{thm}

A proof of this theorem can be found in \cite{hodges93}. When this structure $M$ exists, we will
call it the Fraïssé limit of $\C$.

It is important to note that the uniqueness of the Fraïssé limit only applies to homogeneous
structures. In fact, if we consider the class of finite strict linear orders, we can see it
coincides with not only $\age{\Q}$, but also $\age{\Z}$ and $\age{\N}$. Despite this, all three
of these linear orders are non-isomorphic; this happens as neither $\Z$ nor $\N$ are homogeneous,
as seen in an earlier example.

\subsubsection{Stable theories}%
\label{ssub:stable_theories}

\begin{defn}
  Let $T$ be a complete theory in a countable language. For an infinite cardinal $\kappa$, $T$ is
  $\kappa$-stable if for every every model $M$ of $T$ and subset $A \subseteq M$ with cardinality
  $|A| = \kappa$, then the set of complete $n$-types in $M$ over $A$,  $S_n^M(A)$, has cardinality
  $\kappa$.
\end{defn}

If a theory $T$ is not $\kappa$-stable for any infinite cardinal $\kappa$, then it is called
unstable. Given a model $M$, we say that $M$ is stable (resp. unstable) if the full theory of $M$
is stable (resp. unstable).

The following theorem gives us a characterisation of stability in terms of linear orders, which can
be simpler to reason with, especially when one considers the close relation between linear orders
and interval algebras.

\begin{thm}\label{thm:stable_order_property}
  Let $\theory$ be a complete theory in a countable language, then $\theory$ is stable if and only
  if there exists a formula $\phi(v_1,\dots,v_n,w_1,\dots,w_n)$ and a model $M \models \theory$ with
  a sequence $x_1,x_2,\dots \in M^n$ such that
  \begin{equation*}
    M \models \phi(x_i,x_j) \iff i < j
  \end{equation*}
\end{thm}
\begin{proof}
  {\color{orange} TODO plop a proof down here}
\end{proof}

\begin{exmp}
  A linear order $L$ is stable if and only if it is finite, as a corollary of
  \cref{thm:stable_order_property}.
\end{exmp}

\begin{exmp}
  A complete theory is strongly minimal if for all models $M$, any definable set (with parameters)
  $D \subseteq M$ is either finite or cofinite. This means that unstable theories cannot be
  strongly minimal: Suppose a theory is unstable, then by compactness we may assume there exists
  {\color{orange} TODO try showing a proof here}

  We know from \cite{marker02} that the following are strongly minimal and hence stable:
  \begin{itemize}
    \item The theory of algebraically closed fields in characteristic $p$ in the language of rings:
      all definable sets of an algebraically closed field $k$ are boolean combinations of zero sets
      of polynomials in $k[x]$. Since these zero sets are either finite or all of $k$, then the
      claim of strong minimality follows.
    \item The theory of $\Q$-vector spaces in the language of modules: for a $\Q$-vector space $V$,
      all definable sets $D \subseteq V$ are given by boolean combinations of formulas of the form
      $nx = a$ where $n \in \N$, and $a \in V$. If $a$ is nonzero, such a formula can have at most
      one solution, hence $D$ will have to be either finite or cofinite.
  \end{itemize}
\end{exmp}

\subsubsection{NIP theories}
\label{ssub:nip_theories}

\begin{defn}
  For a complete theory $\theory$, we say that a formula $\phi(x,y)$ has the independence property
  if there is a model $M \models \theory$ and sequences $(a_i)_{i<\omega}$,
  $(b_I)_{I \subseteq \omega}$ in $M$ such that
  \begin{equation*}
    M \models \phi(a_i,b_I) \iff i \in I
  \end{equation*}
\end{defn}

\begin{defn}
  a theory is IP or NIP if blah
\end{defn}

\begin{exmp}
  All stable theories are also NIP since if 
\end{exmp}


\end{document}
