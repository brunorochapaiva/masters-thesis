\section{Introduction}%
\label{sec:introduction}

\subsection{Motivation}%
\label{sub:motivation}

Model theory concerns itself with the connection between first order theories, that is
sets of axioms with universal and existential quantifiers, and the models of these
theories. The compactness theorem and the Löwenheim–Skolem theorems are examples of
significant early results in model theory, both focusing on the existence of models for a
theory: the compactness theorem saying that a model exists for a theory if models
exist for each finite fragment of the theory; and the Löwenheim–Skolem theorems telling
us that if a theory has an infinite model, then there exist models of any cardinality
greater than or equal to the number of symbols in the language.

As a consequence of the Löwenheim–Skolem theorems, we know that there exists no theory
with a unique infinite model up to isomorphism. This raises the question of what can
we say about the number of models a theory has for each cardinality.

\begin{defn}
  For a complete theory $\theory$, we write $I(\theory, \kappa)$ to mean the number of
  models, up to isomorphism, of $\theory$ with cardinality $\kappa$. We call
  $I(\theory, -)$ the spectrum of $\theory$.
\end{defn}

One of the first answers to this question came in the form of Morley's categoricity
theorem, which said that for a countable theory $\theory$, if $I(\theory,\kappa) = 1$ for
some uncountable $\kappa$, then $I(\theory,\lambda) = 1$ for all uncountable $\lambda$
\cite{10.2307/1994188}.Work by Shelah in the 1970s then extended this result to uncountable theories
\cite{Sh:31}, beginning the study of classification theory as a discipline of model theory.

Important notions from stability theory came in the form of stable and unstable theories.
Stable theories were well-behaved enough to limit the number, while in unstable theories
one always has the maximum possible number of models, that is
$I(\theory,\kappa) = 2^\kappa$ for any $\kappa > |\theory|$ \cite{Sh:a}. Being a stable theory
is a very strong condition, hence the study generalisations of stability are important.
One such generalisation, which we will discuss in this report, comes from the
non-independence property, which says that there is no formula able to pick out every
subset of an infinite subset of a model.

Classification theory is not the only active discipline of model theory however. Another
area with a lot of interesting questions comes from the study of homogeneous structures,
where any isomorphism between finite substructures can be extended to an automorphism of
the homogeneous structure. In 1953, Fraïssé showed how to construct such structures
by gluing classes of finite structures together, from which their study followed.

Hence, in the spirit of these disciplines, we will consider the first-order theory of
interval algebras, which were originally introduced by Allen in 1983 to argue about time
qualitatively, and try to find their place in the universe of model
theory, by considering their different models and how well-behaved they are.

\subsection{Report Structure}%
\label{sub:report_structure}

The rest of the report will be structured as follows. In
\cref{sec:background} we summarise the results from Allen's original paper
on interval algebras \cite{allen83}, focusing on the choices that lead to
the 13 relations and the algorithm to infer missing relations from a given
interval network. We also introduce the necessary notions from model theory
such as homogeneous models, and stable/NIP theories. We try to keep this
introduction grounded by working through some examples where possible.
Readers comfortable with these areas may safely skip them in favour of
\cref{sec:axiomatisation} and onwards.

In \cref{sec:axiomatisation} we tackle the question of axiomatisation of
interval algebras. We propose some axioms based on the understanding from
\cref{sec:axiomatisation} and show that these are satisfiable by
constructing models from linear orders. To further justify this
axiomatisation, we will then construct linear orders from interval
algebras.

In \cref{sec:adjunction} we explore further the intervals and points
constructions from \cref{sec:axiomatisation} and show they can be extended to two adjoint
functors between the relevant categories of models. We also give characterisations of
the interval algebras and linear orders for which the unit and counit are isomorphisms.
These characterisations will be expressible as first order sentences, a fact we will use
when studying the classification theory of interval algebras.

Finally, in \cref{sec:model_theory} we study the model theory of interval algebra. This
begins by considering the the class of finite interval algebras and computing its
Fraïssé limit using the machinery developed in \cref{sec:adjunction}. On the topic of
stability, we show that the stable interval algebras are exactly the finite interval
algebras. Following this there is a study of the NIP in interval algebras, for which
we find a big class of NIP algebras as well as an example of an interval algebra with the
IP.

