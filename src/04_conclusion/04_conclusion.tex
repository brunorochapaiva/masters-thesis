\section{Conclusion}
\label{sec:conclusion}

We started this report by establishing a list of axioms for the theory of interval algebras, which
we showed to be satisfiable, containing at least the main class of models we expected. Then,
through further study of the interval construction, we saw it to be a right adjoint. The unit of
this adjunction then meant that every model was a substructure of the intervals over an appropriate
linear order, cementing the relevancy of our axiomatisation.

In attempting to characterise the interval algebras for which the unit was an isomorphism, we found
a formula $\allints$ axiomatising the interval algebras of the form $\inter[L]$. Since the interval
algebra $\inter[L]$ can always be interpretable in $L$, this meant that $\inter[L]$ always has the
NIP.

This is where our work finishes, but there are some possibilities for future work.

First, suppose that we removed the $I \before J$ conjunct from $\allints$ to get
\begin{align*}
  \mostints = & \left(\forall I,\; \forall J,\;
        (I \meets J)     \rightarrow \exists K,\; (I \startedby K)  \land (K \finishes J) \right)\\
    & \land \left(\forall I,\; \forall J,\;
        (I \overlaps J)  \rightarrow \exists K,\; (I \finishedby K) \land (K \starts J)   \right)\\
    & \land \left(\forall I,\; \forall J,\;
        (I \starts J)    \rightarrow \exists K,\; (I \meets K)      \land (K \finishes J) \right)\\
    & \land \left(\forall I,\; \forall J,\;
        (I \finishes J)  \rightarrow \exists K,\; (I \metby K)      \land (K \starts J)   \right)\\
    & \land \left(\forall I,\; \forall J,\;
        (I \contained J) \rightarrow \exists K,\; (I \metby K)      \land (K \starts J)   \right)
\end{align*}
Then it would be interesting to see the type of interval algebras satisfying $\mostints$. We expect
such an interval algebra $A$ to look like a gluing of $\{ \inter[L_i] \}_{i \in I}$ where the
indexing set $I$ is linearly ordered. If this were true, then letting $L$ be the linear order
cosntructed by gluing $\{L_i\}_{i \in I}$, we would expect $A$ to be interpretable in the linear
order $L$ plus some colouring predicates. Since a linear order always has the NIP, regardless of the
number of colouring predicates, this would extend our result about NIP interval algebras.

Another question possibly worth considering further comes from our work done with the interval
construction, which we saw to be a right adjoint functor. From category theory we know that
right adjoints preserve limits, and dually that left adjoints preserve colimits. Work by
Caramello \cite{caramello08} shows how to realise the Fraïssé limit of a Fraïssé class as a
colimit in the language of category theory. So here we have a functor, $\inter$, preserving quite
a complicated colimit. This could hint at a couple of options: it might be the case that $\inter$ is
also a left adjoint functor, although if this is the case it would have to be some new functor
distinct from $\points$. Alternatively, there could be something specific about Fraïssé limits
which meant they should be preserved by right adjoints. In either case, it would be interesting to
see what develops.